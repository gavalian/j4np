\documentclass[aps,prl,preprint,12pt]{revtex4}
\usepackage{CJK}
\usepackage{graphicx}
%\usepackage{authblk}

\begin{document}

\begin{CJK*}{GB}{}
\title{CLAS12 Track Reconstruction with Artificial Intelligence}

\author{Gagik Gavalian}
\affiliation{Jefferson Lab, Newport News, VA, USA}
\author{Polykarpos Thomadakis}
\author{Angelos Angelopoulos}
\affiliation{CRTC, Department of Computer Science, Old Dominion University, Norfolk, VA, USA}
\author{Raffaella De Vita}
\author{Veronique Ziegler}
\affiliation{Jefferson Lab, Newport News, VA 23606, USA}
\author{Nikos Chrisochoides}
\affiliation{CRTC, Department of Computer Science, Old Dominion University, Norfolk, VA, USA}

\begin{abstract}
  In this article we describe the implementation of Artificial Intelligence models in track reconstruction
  software for the CLAS12 detector at Jefferson Lab. The Artificial Intelligence based approach resulted 
  in improved track reconstruction efficiency in high luminosity experimental conditions.  The track
 reconstruction efficiency increased by $10-12\%$ for single particle, and statistics in multi-particle physics 
 reactions increased by $15\%-35\%$ depending on the number of particles in the reaction. The implementation 
 of artificial intelligence in the workflow also resulted in a speedup of the tracking by $35\%$. 
 % we estimated these 
 %improvements are equivalent of more than $\$4$M  annual savings for realization of the CLAS12 physics program.
\end{abstract}

\maketitle
\end{CJK*}

\section{Introduction}
\indent

Nuclear Physics experiments have become increasingly complex over the past decades, with more complex detector 
systems and higher luminosities. In emerging experiments where data collection rates are higher, there is a need for 
new approaches for data processing that can improve data reconstruction accuracy and speed. New developments in 
the Artificial Intelligence (AI) field present promising alternatives to conventional algorithms for data processing. 
Machine Learning (ML) algorithms are being employed in various stages of experimental data processing, such as: 
the detector data reconstruction, particle identification, detector simulations and physics analysis. 

In this paper we present the implementation of machine learning models into the CLAS12 charged-particle track 
reconstruction software. Detailed analysis of the reconstruction performance are presented, comparing track 
reconstruction efficiency and speed improvements to conventional algorithms.

%\section{CLAS12 experiments}

%To estimate the statistical accuracy of proposed experiments calculation are made to derive 
%best experimental conditions to achieve physics goals. One of the large experiments running 
%on CLAS12 detector is Deeply Virtual Compton Scattering (DVCS) rung group (RGA). The 
%estimations for experimental time were calculated assuming $100\%$ tracking efficiency at 
%$90~nA$ (equal to $L=2\times10^{35}~cm^{-2}sec^{-1}$) and was estimated that $80$ days
%of experimental time is needed to achieve the physics goals. However, the conventional tracking 
%algorithm does not perform as expected and can provide only $90\%$ tracking efficiency at $45~nA$
%incident beam energy. Currently experiments run at $45~nA$ (equivalent to 
%$L=1\times 10^{35}~cm^{-2}sec^{-1}$) which will only achieve half of desired statistical requirements
%of experiment. With use of Artificial Intelligence we intend reduce the discrepancy between experimental
%time and physics goals.

%************************************
%-------- Included Chapters
\section{Particle Tracking}

The CLAS12\cite{Burkert:2020akg} detector is built around a six-coil toroidal magnet which divides the active detection into six azimuthal regions, called "sectors". Each sector is equipped with three regions of drift chambers. Each region consists of two chambers (caller super-layers), with each of them having 6 layers of wires. Each layer of wires in super-layer contains 112 wires making a super-layer a 6x112 wire matrix. The schematic view of one super-layer is shown on Figure~\ref{dc:side_view} (right panel).

\begin{figure}[!ht]
\begin{center}
 \includegraphics[width=3.5in]{images/dc_side_view.pdf}
 \includegraphics[width=2.5in]{images/image-29.jpg}
\caption {Side and front view of drift chambers. a) the layout for one sector showing three regions of drift chamber. Z-axis is direction of incoming beam. b) diagram of wire directions in each super layer.}
 \label{dc:side_view}
 \end{center}
\end{figure}

Particle that originates at interaction vertex in forward direction that pass through all three regions of the drift chamber in given sector are reconstructed by tracking algorithms. First adjacent  wires in each super-layer with a signal are grouped together into clusters (called segments), then track candidates are constructed by forming 6 segment combinations (one per each super-layer). Each track candidate is validated by tracking algorithm by performing a polynomial fit to determine if the candidate can be a real track. Once track candidates are isolated they are further refined using Kalman filter to measure particle parameters (such as momentum and angles). 
There are many experiments conducted with CLAS12 detector which have different running conditions (i.e. different targets and beam currents).
The changing beam energies and beam intensities (current) affect the amount of background that is detected in each detector components. For drift chambers the added background can be in a form of random hits that can be isolated by noise reducing algorithm, and can be in form of segments that do not belong to a track. With increased luminosity (beam current and target combination) number of combinatorics and the candidates to consider significantly increases, and this leads to decreased efficiency in track reconstruction. Recent studies done with experimental data and simulation showed that the tracking efficiency decreases with a rate of $0.44\%$ per nA of beam current. This leads to tracking efficiency at standard experimental running conditions (which is $45~nA$ electron beam incident on $40~cm$ liquid hydrogen target) is $\approx 80\%$.
The decrease of the efficiency of tracking what prevents experiments to run at higher beam currents (interaction rates), and increasing tracking efficiency will allow experiments to run for shorter time to collect desired statistics for physics results. 

\section{Machine Learning}

The Machine Learning approach was to teach the neural network to recognize good tracks
 from a large list of candidates. 
 Two neural networks were developed: a classifier network that can identify good track from  6 segment
 track candidates  and an auto-encoder that can take a list of 5 segment tracks and make them into 6 segment 
 track candidates by adding a pseudo-segment. The composed 6 super-layer track candidates (with pseudo-cluster)
 can be processed with classifier network to identify and isolate "good" track candidates.

 %The implementation consisted of two neural networks: a classifier network
 %that can identify good track from  6 segment track candidates  and a auto-encoder that can take a list of
 %5 segment tracks and make them into 6 segment track candidates by adding a pseudo-segment.
 
 \subsection{Track Classifier}
 
 To determine what type of architecture works best with CLAS12 drift chamber data we investigated different 
 types of neural network including Convolutional Neural Network (CNN) , Extremely Randomized Trees (ERT) and 
 Multi-Layer Perceptron (MLP) \cite{Gavalian:2020oxg}. The study showed that Multi-Layer Perceptron was best suited for 
 CLAS12 reconstruction needs (based on inference speed and accuracy). The implemented architecture is shown on 
 Figure~\ref{mlp:architecture}, where an input layer with 6 nodes is used (each node representing average wire position 
 of the segment in super-layer) and 3 output nodes for classes "positive track", " negative track" and "false track".
 
 \begin{figure}[!ht]
\begin{center}
 \includegraphics[width=3.5in]{images/Multilayer-Perceptron.jpg}
% \includegraphics[width=2.5in]{images/image-29.jpg}
\caption {Architecture of Multi-Layer Preceptron.}
 \label{mlp:architecture}
 \end{center}
\end{figure}

The network is trained using results from processed data where the conventional algorithm has already 
identified good tracks (with a cut on $\chi^2$ to select very good quality tracks), which are fed to the network with
their respective labels (i.e. positive or negative tracks). For false tracks a combination of segments (6 segments 
forming a track candidate) is chosen that was not identified as a track by conventional algorithm.
 
 \subsection{Corruption Auto-Encoder}
 
The second neural network was developed to fix the corruption in possible track candidates due to 
inefficiencies of drift chambers. This network will be used to identify track candidates which have one of the segments
missing. We used auto-encoder type of neural network to implement feature fixing neural network \cite{Gavalian:2020xmc}. 
The structure of the network can be seen on Figure~\ref{autoencoder:architecture}, with 6 input nodes and six output nodes.

 \begin{figure}[!ht]
\begin{center}
 \includegraphics[width=2.0in]{images/auto_encoder.png}
\includegraphics[width=2.0in]{images/auto_encoder_result_2d.pdf}
\includegraphics[width=2.0in]{images/auto_encoder_result_1d.pdf}
\caption {Architecture of corruption fixing Auto-Encoder.}
 \label{autoencoder:architecture}
 \end{center}
\end{figure}

To train the network good track candidates reconstructed by conventional tracking algorithm are used (same sample 
as for track classifier network training). The output for the network was set to the good track parameters (where all 6 
segments have non-zero values) and the input was modified by setting one of the nodes (randomly) to zero. The network
learns to fix the node containing zero by assigning it a value based on the other 5 segment values. The test results 
of the trained network are shown on Figure~\ref{autoencoder:architecture}, where the difference between true value 
of the segment position and reconstructed by network position is plotted, showing reconstruction accuracy of $0.36$ wires.
On Figure~\ref{autoencoder:architecture} this difference is shown vs the super-layer number which was corrupted in the input, 
as can be seen the performance of the network is uniform across all super-layers.

%\subsection{Implementation in reconstruction software}
%The CLAS12 track reconstruction software consist of several parts. The first stage of the process is to isolate clusters
%from the hits in drift chambers (called clustering service) . Once the clusters are isolated track candidates are formed from all combinations 
%of 6 segments. The track candidates are analyzed to determine which good candidates, from remaining segments 
%combinations of 5 segment tracks are constructed and these are also fitted to determine which ones are potential good tracks.
%The later module (hit based tracking) determines good track candidates based only on hit positions of the track candidates (no timing
%informations is used at this level). At the later stages of tracking code (time based tracking), chosen good track candidates are further refined by use of Kalman filter by using timing information from each of the sensors (drift chamber wires). By the time the reconstruction code reaches time based
%tracking stage the track candidates are already defined. 

%\begin{figure}[!ht]
%\begin{center}
% \includegraphics[width=6.0in]{images/recon_diagram.png}
%\caption {Architecture of corruption fixing Auto-Encoder.}
% \label{recon:diagram}
% \end{center}
%\end{figure}

%In order to implement our neural network into reconstruction workflow we designed two parallel branches in reconstruction code where we run 
%two algorithms to identify good tracks from track candidate lists, one based on conventional algorithm and second based on neural network.
%Both algorithms store their track suggestions in separate data structures, and pass them to the next stage where track parameters are reconstructed by 
%conventional tracking algorithm using Kalman filter. This approach let's us have two parallel outputs from tracking code in order to compare performance of
%each of the methods.

\section{Implementation of the Neural Network in CLAS12 software}

The models described in the previous sections were implemented in the CLAS12 tracking software. 

\subsection{Track Identification Workflow}

 Track identification consists of two phases, programmed to be done in two passes. In the first pass 
 over the data, signals from each sector of drift chambers are analyzed to create a track candidate list, 
 each consisting of 6 segments. The resulting track candidates are evaluated by the classifier neural 
 network and are assigned a probability of being either a positive or negative track. The list of track 
 candidates is sorted by probability and passed to another algorithm that is responsible for removing 
 tracks that have a lower probability of being a "good" track and have clusters that are shared with a 
 higher-probability candidate. 

In this procedure, the algorithm iterates over-track candidates sorted according to the probability of 
being a good track. Iteration starts at position number 2 and runs to the end of the list. Candidates that 
share a cluster with candidates at position number 1 are removed from the list. Track candidate at position 
1 is moved from the track candidate list into the identified track list. This procedure is repeated until 
there are no track candidates left in the candidate list.

 \begin{figure}[!h]
\begin{center}
 \includegraphics[angle=90,width=1.1in]{images/iden_6_sl.pdf}
  \includegraphics[angle=90,width=1.1in]{images/iden_5_sl_a.pdf}
    \includegraphics[angle=90,width=1.1in]{images/iden_5_sl_b.pdf}
      \includegraphics[angle=90,width=1.1in]{images/iden_5_sl_c.pdf}
            \includegraphics[angle=90,width=1.1in]{images/iden_5_sl_d.pdf}
            
\caption {Stages of Neural Network track identification procedure. 1) identifying 6 super-layer tracks. 2) 
removing all hits belonging to an identified track and constructing 5 super-layer track candidates. 
3) generating pseudo-clusters for 5 super-layer track candidates using corruption fixing auto-encoder. 
4) identify good track candidates from the list of 6 super-layer (one of the super-layers is a pseudo-cluster) 
track candidates. 5) isolate both identified (6 super-layer and 5 super-layer) tracks  for further fitting with 
Kalman-Filter.}
 \label{network:procedure}
 \end{center}
\end{figure}

The second stage of track identification starts by constructing a list of track candidates with combinations 
of 5 clusters out of 6 from all existing clusters (one per super-layer). The candidates that share a cluster with 
tracks identified at the first stage of classification are removed from the list. For each track candidate with a 
missing cluster in one of the super-layers, a pseudo-cluster is generated using the Corruption Auto-Encoder 
Network and the missing super-layer cluster are assigned the inferred value, hence turning all track candidates 
to 6 cluster track candidates. The cured (or fixed) track candidate list is finally passed to the track classifier 
module described above, which evaluates the list isolating candidates with the highest probability of being a 
good track. 

\subsection{Implementation in reconstruction software}

The CLAS12 reconstruction software framework is a Service Oriented Architecture platform implemented in Java (CLARA~\cite{Gyurjyan:2011zz}).
The reconstruction software consists of several microservices, each responsible for processing data from one
detector \cite{Ziegler:2020gsr}. The reconstruction procedure for some of the detector components can also 
be broken down into smaller logical microservices to add some flexibility in changing the implementation of 
the small parts and provide alternative reconstruction procedures for some of the components. Reconstruction 
of tracks in drift chambers is a complex task and consists of several parts.

The first stage of the process, called clustering service, is to isolate clusters from the hits in drift chambers. 
Once the clusters are isolated, track candidates are formed from all combinations of 6 segments. Once 6 
segment tracks have been identified, the remaining segments are then used to form 5 segment combinations. 
These two steps are known as track-finding or seeding. Both 6 and 5 segment candidates are fitted defining 
the hit positions from the wire coordinates (hit-based tracking), resulting in the first list of reconstructed tracks.

In a second stage of the tracking code (time-based tracking), the tracks identified at the previous stage are 
refitted with a Kalman Filter algorithm, which uses drift time information to refine the hit positions. This 
produces the final track list.

\begin{figure}[!ht]
\begin{center}
 \includegraphics[width=6.0in]{images/CLARA_AI_diagram.png}
\caption {Diagram of the tracking workflow with Artificial Intelligence included. The Workflow is split into 
two parallel branches, one where track-finding is done by the conventional algorithm, and one where only 
AI-isolated tracks are fed to the following stages.}
 \label{recon:diagram}
 \end{center}
\end{figure}

In order to implement our neural network into the reconstruction workflow, we designed two parallel 
branches in the reconstruction code where we run two algorithms to identify good tracks from the 
candidate lists, one based on the conventional algorithm and one based on the neural network. The two 
algorithms store their track suggestions in separate data structures and pass them to the next stage where 
track parameters are reconstructed by a conventional tracking algorithm, first by hit-based fitting and then 
using the Kalman filter. This approach lets us have two parallel outputs from the tracking code that enables 
a detailed comparison of the performance of each method.

\subsection{Software Packages}

As mentioned above, CLAS12 reconstruction is implemented in Java. The final implementation of the  
track classifier was therefore in Java for easy integration into the reconstruction workflow. The initial tests 
and prototyping were done using the Keras/TensorFlow~\cite{keras-website} (python) package. The final 
implementation was based on the DeepNetts \cite{Sevarac.Z} community edition library (in native Java) 
used for both the track candidate classifier and corruption-recovery auto-encoder. DeepNetts is a lightweight
 library with minimal dependencies, which makes it ideal for providing portable code that can be used on a variety 
 of platforms without the need of installing a large number of platform-dependent packages (like in the case of
  keras/tensorflow). The inference procedure was implemented using Efficient Java Matrix Library (EJML)~\cite{ejml:2021}, 
  which is optimized for speed and is thread-safe, matching the requirements of the CLAS12 reconstruction software.

The analysis and data visualization for this article was done using GROOT data visualization package~\cite{groot-github} 
developed for CLAS12 software infrastructure (in Java) and is included in the Java data analysis library for high energy 
physics Jas4pp~\cite{Chekanov:2020bja}.

\section{Analysis of Track Reconstruction with AI}

After the implementation of track identification service in CLAS12 reconstruction software the outputs
from conventional tracking algorithm and artificial intelligence assisted tracking algorithm were analyzed
event by event to ascertain improvements of tracking. 
 
 \subsection{Particle Reconstruction efficiency}
 
 The Neural Network for track classification was trained on experimental data after it was processed with conventional tracking 
 reconstruction. Track that have "good" fit quality and were tracked back to the target were used as a training sample for both 
 MLP classifier and Auto-Encoder corruption fixing network. For detailer analysis of tracking reconstruction performance with and without 
 assistance from artificial intelligence we processed one run at nominal luminosity (45 nA) compare performances.
 
 %The efficiency of track reconstruction was obtained for separate track topologies (6 super-layer and 5 super-layer).
 \begin{figure}[!h]
\begin{center}
% \includegraphics[width=2.0in]{images/pos_theta_5SL.png}
  \includegraphics[width=5.0in]{images/summary_5SL_6SL_neg.pdf}
\caption { }
 \label{track:efficiency}
 \end{center}
\end{figure}

The results are shown on Figure~\ref{track:efficiency}, where dependence of number of reconstructed negatively charged 
 tracks are shown as a function of particle momentum (top row), polar angle in laboratory frame (bottom row) and interaction
 vertex (middle row). The reconstructed distributions from conventional tracking are plotted with filled histograms and the
 tracks reconstructed by using assistance from AI are plotted with solid lines. As can be seen from the figure there is very big gain 
 in number of reconstructed tracks with 5 super-layer configuration compared to full 6 super-layer tracks. Typically for nominal 
 45 nA experimental data increase track efficiency for 6 super-layers tracks averages about $3\%-6\%$, while for 5 super-layer
 tracks the increase is in the order of $70\%-120\%$. In normal data reconstruction tracks that are identified with 5 super-layers
 usually comprise about $10\%$ of all reconstructed tracks, and significant increase in identification of such tracks leads to 
 overall tracking efficiency increase of $12\%-15\%$. 
 
 \begin{table}[!h]
 \begin{center}
 \begin{tabular}{|l|c|c|c|c|}
 \hline
 Track Configuration & Conventional & AI Assisted & Gain & Relative \\
 \hline
 \hline
 6 Super-Layer & 242,145 & 256,175 & 14030 & 1.0579 \\
 5 Super-Layer & 24,155 & 52,839 & 28684 & 2.1875 \\
 All & 267,339 & 309,058 & 51719 & 1.1561 \\
 \hline
 \end{tabular}
 \end{center}
 \caption{Summary of reconstructed tracks and gain with assistance from Artificial Intelligence.}
 \label{tbl:summary}
 \end{table}
 
The comparison of 5 super-layer and 6 super-layer track statistics and their relative gain is summarized in Table~\ref{tbl:summary}.
As can be seen from the table the gain in only 6 super-layer tracks is about $5.7\%$ but with significant gain in 5 super-layer tracks 
the overall gain in reconstructed tracks elevates to $>15\%$. These results are intuitive since track candidates composed of 5
super-layers with the same number of clusters in each super-layer are significantly higher than 6 super-layer track candidates, and 
it our tests AI performs better in choosing right combination with increasing combinatorics.
 
\subsection{Luminosity Dependence}

Track reconstruction efficiency increased with AI assisted tracking, since AI can better identify tracks
from pool of candidates. One would expect that if the number of combinations decrease efficiency 
of conventional track selection algorithm should approach efficiency of AI assisted track identification,
similarly, when number of combination increases the advantage of AI over the conventional algorithm should
increase. Based on this we expect AI to perform better in higher background settings. To evaluate AI assisted
tracking efficiency dependence on background we analyzed several different runs that were taken in different 
conditions (i.e. beam current) ranging from $5~nA$ to $70~nA$. To measure tracking efficiency we first calculated
number of electrons ($N_e$) detected in the data sample analyzed (typically one run) and then number of positive and negative
 hadrons that were detected with the electron inclusively ($N_{h^+e}$ and $N_{h^-e}$ respectively).

Then the efficiency for the data set was calculated as:

\begin{equation}
L_t^+ = \frac{N_{h^+e}}{N_e} , L_t^- = \frac{N_{h^-e}}{N_e} 
\end{equation}

where $L_t^+$ is the efficiency of positive particles and $L_t^-$ is the efficiency of negatively charged particles respectively. 
In order to estimate the charged particle reconstruction efficiency as a function of the beam current, the multiplicity, $L_t^{+/-}$, is fitted with a linear function:
\begin{equation}
L_t^{+/-} = a + c\times I 
\end{equation}

Here $a$ and $c$ are the fit parameters and $I$ is the beam current. Then it was assumed that the reconstruction efficiency, $E=1$ at $I=0$ nA:

\begin{equation}
E^{+/-} = 1 + b \times I 
\end{equation}

with $b=\frac{c}{a}$. The slope parameter b is the rate of the reconstruction inefficiency as a function of the beam current \cite{Stepanyan:2020bg}.
%The all points were 
%fitted with linear function $L=a+bx$, where $a$ is the intercept 
 
 \begin{figure}[!ht]
\begin{center}
 \includegraphics[width=3.0in]{images/lumi_scan_positive.pdf}
 \includegraphics[width=3.0in]{images/lumi_scan_negative.pdf}
\caption {Tracking efficiency for positively and negatively charged particles as a function of beam current (luminosity).  Conventional algorithm 
track reconstruction efficiency (diamonds) is compared to AI assisted track reconstruction efficiency (circles). }
 \label{lumi:scan}
 \end{center}
\end{figure}

The comparison of tracking efficiency as a function of beam current (luminosity) can be seen on Figure~\ref{lumi:scan} where $E^{+/-}$ are shown for positively and negatively charged particles separately. As can be seen from the figure AI assisted tracking performs significantly better for any given luminosity (beam current) and the decrease of efficiency is much slower as function of luminosity, $0.22\%$ per nA versus $0.40\%$ per nA for conventional tracking. This is expected and consistent with assumption that with increased combinatorial background (increased number of track candidates to consider) , AI performs better in choosing best track candidate. We established that AI assisted
tracking leads to more tracks reconstructed for any given beam current setting, next thing to check is what is the impact of increase track reconstruction
efficiency on physics analysis, and if there is increase in physics outcome for the CLAS12 experimental setup.

\subsection{Physics Impact}

To measure practical implications of track reconstruction efficiency improvement on physics outcome we analyzed 
two event topologies with two particle and three particles in the final state respectively. The data for analysis were 
taken with $10.5~GeV$ electron beam incident on $20~cm$ liquid hydrogen target, with the beam current of $45~nA$
(typical for CLAS12 experimental running). We selected events where an electron was detected in the forward detector 
then isolated events where there was an additional negatively charge pion ($\pi^-$) along with electron and no other 
charged particle, and the second topology required two pions along with electron, one positively charged and one 
negatively charged. The two chosen topologies are denoted by $H(e,e'\pi^-X)$ and $H(e,e'\pi^+\pi^-X)$. In both cases 
there is a visible peak of a missing proton that we can use to measure impact of efficiency on physics outcome. 

 \begin{figure}[!ht]
\begin{center}
 \includegraphics[width=6.0in]{images/physics_results.pdf}
\caption {Architecture of corruption fixing Auto-Encoder.}
 \label{physics:outcome}
 \end{center}
\end{figure}

The distributions of missing mass for both final state topologies are shown on Figure~\ref{physics:outcome}, the plots 
on the top row are missing mass of $H(e,e'\pi^-X)$ and $H(e,e'\pi^+\pi^-X)$, where the filled histogram is calculated from 
particles reconstructed by conventional tracking algorithm, and the histogram with black outline are same distributions 
calculated from particles that were reconstructed using suggestion from Artificial Intelligence. As can be seen from the figure 
there is significant increase in number of events under the missing proton peak (at mass value $0.938~MeV$) for AI assisted
tracking. The ratios of two histograms (AI assisted divided by Conventional) can be seen on the bottom row of 
Figure~\ref{physics:outcome}. As can be seen from the figure the increase in statistics is uniform over the whole range of the 
missing mass indicating no systematic abnormalities for AI assisted tracking. The ratio also indicates that there is an increase 
for number of event under the peak for proton, about $15\%$ for $H(e,e'\pi^-X)$ final state and $30-35\%$ for the $H(e,e'\pi^+\pi^-X)$
final state. Further studies show that increase in statistics for different final states increase with increase of beam current (luminosity) 
which is consistent with our studies of increased efficiency of single particle reconstruction. 


\section{Summary}

In this paper we investigated results of analysis of experimental data from CLAS12 detector reconstructed with assistance of Artificial Intelligence
to identify tracks from the hits in Drift Chambers. This work is based on two Neural Networks developed to classify track candidates from
given cluster combinations \cite{Gavalian:2020oxg} and to identify missing cluster positions in tracks that do not have complete 6 cluster configuration \cite{Gavalian:2020xmc}. After implementing these networks into CLAS12 reconstruction workflow, the AI was able to identify "good" track candidates 
and pass them to tracking code to be analyzed parallel to conventional algorithm that chooses "good" track candidates iteratively considering all possible combinations. 
Our studies showed that AI assisted tracking performs better than conventional track identification algorithm, and leads to track reconstruction efficiency increase of $15\%$ for nominal experimental running conditions (beam current 45 nA). The AI also performs better with increasing background (i.e. with increased incident beam current) and improves the efficiency loss from $0.44\%$ per nA to $0.24\%$ per nA.
This increased track reconstruction (identification) efficiency directly impacts the outcome of physics analysis where it increases statistics 
for physics reactions for $15\%-35\%$ depending on how many particles are detected in the final state and the topology of the reaction. This has big implication on experimental running conditions, since with increased efficiency required statistical significance of the experiment can be reached in shorter time by running at higher beam current (luminosity). Already collected experimental data be re-processed with the AI assisted tracking
code which can increase the statistics for analyzed data up to $35\%$. Both, future experiments and already completed experiment will benefit 
from this novel development.

Another important outcome of this development was reduction in data processing times, since track candidates were identified by Artificial Intelligence 
there were fewer marginal quality tracks picked to be analyzed and then later dropped due to non convergence of Kalman-Filter, and this leads to tracking code speed-up of $35\%$.

Overall we identified that Artificial Intelligence approach to assisting tracking codes is a good approach, it leads to improvements in code speed and 
efficiency of track reconstruction. Another important aspect of using Artificial Intelligence is that is leads to very small and simple code base, comprised of composing track candidates and feeding them to neural network, and what's also important it keeps improving with constant training with new data.
We intend to continue this development in extending the approach to other tracking detectors of CLAS12, and possibly try to adopt  our approach for other experimental detector setups at Jefferson Lab.



\section{Acknowledgments}

This material is based upon work supported by the U.S. Department of Energy, Office of Science, Office of Nuclear 
Physics under contract DE-AC05-06OR23177, and NSF grant no. CCF-1439079 and the Richard T. Cheng Endowment. 
This  work  was  performed  using  the  Turing  and  Wahab computing clusters at Old Dominion University.
 
\newpage
\bibliography{references}
\bibliographystyle{ieeetr}



\end{document}
