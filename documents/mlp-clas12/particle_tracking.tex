\section{Charged Particle Tracking}

The CLAS12~\cite{Burkert:2020akg} forward detector is built around a six-coil toroidal magnet 
which divides the active detection area into six azimuthal regions, called ``sectors''. Each sector is 
equipped with three regions of drift chambers~\cite{Mestayer:2020saf} designed to detect charged 
particles produced by the interaction of an electron beam with a target. Each region consists of two 
chambers (called super-layers), each of them having 6 layers of wires. Each layer  in a super-layer 
contains 112 signal wires, making a super-layer a 6x112 cell matrix. The schematic view of one region 
is shown on Figure~\ref{dc:side_view} (right panel).

\begin{figure}[!ht]
\begin{center}
 \includegraphics[width=3.1in]{images/dc_diagram.pdf}
 \includegraphics[width=3in]{images/region_2_diagram.pdf}
\caption {Schematic view of signals generated in the drift chambers when a particle passes through. 
The segments in each super-layer are shown along the trajectory of the track (left panel), and view 
of the activated cells in the two super-layers of one region along the track trajectory (right panel).}
 \label{dc:side_view}
 \end{center}
\end{figure}

Particles that originate at the interaction vertex travel through magnetic field and pass through all 
three regions of the drift chamber in a given sector are reconstructed by tracking algorithms. First, 
in each super-layer adjacent  with a signal are grouped together into clusters (called segments), 
shown on Figure~\ref{dc:side_view}. The positions of these segments (clusters) in each super-layer 
are used to fit the track trajectory to derive initial parameters, such as momentum and direction. 
After the initial selection, good track candidates are passed through Kalman~\cite{Kalman1960} 
filter to further refine measured parameters.

\begin{figure}[!ht]
\begin{center}
 \includegraphics[width=6.2in]{images/figure_dc_examples.pdf}
\caption {Example of signals in drift chambers for few events. Each plot represents one sector with 
a 36x112 matrix of wires. Background hits (in gray) are shown along with the hits of reconstructed 
identified tracks (in red). Dashed lines represent boundaries between super-layers.}
 \label{dc:events_sector}
 \end{center}
\end{figure}

For each beam-target interaction or ``event'', drift chambers produce many segments, some belonging 
to a track and some to background, or partial trajectories of low momentum tracks. In Figure~\ref{dc:events_sector} 
drift chamber signals in one sector are shown for four different  events, in each sector data are hits 
represented as a 36x112 matrix (36 layers and 112 wires per layer), showing all hits including those that 
were determined to be part of a track. 

Due to inefficiencies in the drift chambers, it is possible to have one missing segment along the trajectory 
of the particle, and the track has to be reconstructed using only 5 segments. An example of a 5-segment 
track is shown on Figure~\ref{dc:events_sector} c), where super-layer 3 does not have any segment detected. 
For these types of tracks, candidates have to be identified from a large number of combinatorics consisting 
of all combinations of clusters that form a  5-segment candidates. 

Tracking is computationally extensive and makes up $80\%-90\%$ (depending on background conditions 
and track multiplicity) of the total CLAS12 event processing time. The procedure of finding tracks from a 
list of track candidates is where we found AI can provide real benefits. Such benefits include: improved 
accuracy in identifying good tracks, and improved data processing speed by significantly reducing the 
number of candidates that have to go through initial fitting and then through Kalman filter. AI can also 
help in identifying 5 super-layer track candidates.
