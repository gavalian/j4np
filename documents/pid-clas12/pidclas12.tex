\documentclass[preprint,12pt]{elsarticle}
\usepackage{graphicx}
\usepackage[margin=1.0in]{geometry}
\usepackage{color, colortbl}
\usepackage{hyperref}
\usepackage{float}
% \usepackage[affil-it]{authblk}
\usepackage{subcaption}
\newcommand{\note}[1]{\textcolor{blue}{#1}}
\definecolor{LightCyan}{rgb}{0.88,1,1}
\definecolor{LightRose}{rgb}{1,0.88,0.88}
\definecolor{LightGreen}{rgb}{0.88,1,0.88}

\title{Particle Identification in CLAS12 using Artificial Intelligence}
\author[1]{Gagik Gavalian}
%\author[2]{Polykarpos Thomadakis }
%\author[2]{Angelos Angelopoulos}
%\author[1]{Veronique Ziegler}
%\author[1]{Raffella De Vita}
%\author[2]{Nikos Chrisochoides}

% \affiliation[1]{organization={CRTC, Department of Computer Science, Old Dominion University}, city={Norfolk, VA}, country={USA}}
\address[1]{Jefferson Lab, Newport News, VA, USA}
\address[2]{CRTC, Department of Computer Science, Old Dominion University, Norfolk, VA, USA}
%Authored by Jefferson Science Associates, LLC under U.S. DOE Contract No. DE-AC05-06OR23177. The U.S. Government retains a non-exclusive, paid-up, irrevocable, world-wide license to publish or reproduce this manuscript for U.S. Government purposes.

%\fntext[fn1]{Authors contributed equally.}
%\fntext[fn2]{Correspoding author, \textit{gavalian@jlab.org}}


\begin{document}

%\begin{titlepage}

\begin{abstract}

  In this article we describe implementation of Artificial Intelligence models in track reconstruction software for CLAS12 detector at Jefferson Lab.
 The Artificial Intelligence based approach resulted in improved track reconstruction efficiency in high luminosity experimental conditions.  The track
 reconstruction efficiency increased by $15\%$ for single particle, and statistics in multi-particle physics reactions increased by $15\%-35\%$ depending 
 on number of particles in the reaction. Implementation of artificial intelligence in workflow also resulted in code speedup of $35\%$.
\end{abstract}
%\end{titlepage}
\maketitle


%\section{Introduction}

Over the decades, high-energy experiments, such as those conducted at particle accelerators like CERN’s Large Hadron Collider (LHC), have seen an exponential increase in data production. Early particle physics experiments in the mid-20th century relied on photographic plates and bubble chambers, which generated manageable amounts of data for manual analysis. As technology advanced, detectors became more sophisticated, capturing finer details of particle collisions across multiple dimensions, from spatial tracks to energy signatures. This evolution enabled scientists to probe deeper into the structure of matter and the fundamental forces of nature but came with an enormous uptick in data volume.

Modern experiments produce petabytes of data annually, thanks to the use of advanced digital detectors and high-frequency collision rates. For instance, the LHC can generate up to one billion proton-proton collisions per second during its operations. Each collision results in complex, high-dimensional data that must be recorded, processed, and analyzed. However, the vast majority of these collisions are routine and unremarkable, reflecting well-known physics processes. Only a tiny fraction contains the rare and novel events that could reveal new particles or phenomena, such as the discovery of the Higgs boson in 2012.

The challenge lies in efficiently processing this deluge of data to identify and retain only the relevant information while discarding the rest. This is achieved through a multi-layered system of data selection. First, hardware-based triggers operate in real-time to reduce data rates by orders of magnitude, selecting events based on basic characteristics like energy thresholds. Then, software-based algorithms provide more refined filtering by analyzing the remaining data for specific patterns or anomalies. Despite these strategies, the volume of "useful" data still reaches hundreds of terabytes, requiring vast computing resources and distributed networks to store and process the information.

Another key challenge is ensuring that the data reduction process does not inadvertently discard valuable signals. Designing selection algorithms requires balancing sensitivity to rare events with the need to filter out noise effectively. Advances in machine learning have become increasingly important in addressing this problem. Sophisticated models can analyze high-dimensional data for subtle correlations and anomalies, improving the ability to identify rare events. However, the computational demands of such models add another layer of complexity, requiring extensive computing power, energy, and expertise.

The rise of big data in high-energy physics not only highlights the field’s technological achievements but also underscores the growing need for innovative solutions in data management and analysis. Collaboration between physicists, computer scientists, and engineers will remain critical in addressing these challenges and unlocking the secrets hidden in the vast streams of experimental data.


\section{Calorimeter}

A primary goal of the CLAS12 physics program is to study internal dynamics of the nucleon . 
These experiments require accurate kinematical analysis of neutral and charged particles at high momentum. 
In particular, all CLAS12 electro-production experiments require the efficient detection and reliable identification 
of energetic electrons, photons, and neutrons using the forward electromagnetic calorimeter (ECAL).

One of the primary usages of ECAL system is separating electrons from other particles, like pions, using 
energy deposited in the calorimeters. To accommodate hexagonal design of CLAS12 detector ECAL is using 
triangular hodoscope layout. The scintillating layers have three alternating stereo readout planes named U,V and W, 
which are interleaved with layers of lead as shown on Figure~\ref{clas12:ecal}

\begin{figure}[!ht]
\begin{center}
 \includegraphics[width=3.in]{images/calorimeter_layers.png}
 \includegraphics[width=2in]{images/ecal_view.png}
\caption { CLAS12 Electromagnetic Calorimeter structure description.}
 \label{clas12:ecal}
 \end{center}
\end{figure}

When particle enters the calorimeter it leaves signal in each of the layers (U,V and W), and is readout independently.
For each layer a cluster (called peak) is constructed by grouping adjacent hit strips and peaks from all three sides are combined into a cluster if they intersect in one point on the surface. A typical cluster is shown on Figure~\ref{clas12:ecal}.



\section{Acknowledgments}
This material is based upon work supported by the U.S. Department of Energy, Office of Science, Office of Nuclear Physics under contract DE-AC05-06OR23177, and NSF grant no. CCF-1439079 and the Richard T. Cheng Endowment. 
 
\newpage
\bibliography{references}
\bibliographystyle{ieeetr}

\end{document}
