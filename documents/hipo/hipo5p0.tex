\documentclass[preprint,12pt]{elsarticle}
\usepackage{graphicx}
\usepackage[margin=1.0in]{geometry}
\usepackage{color, colortbl}
\usepackage{hyperref}
\usepackage{float}
\usepackage{lineno}
\usepackage{tabularx} % For specifying table width
\usepackage{array}    % For additional column formatting options
%\linenumbers
% \usepackage[affil-it]{authblk}
\usepackage{subcaption}
\usepackage{listings}

\newcommand{\note}[1]{\textcolor{blue}{#1}}
\definecolor{LightCyan}{rgb}{0.88,1,1}
\definecolor{LightRose}{rgb}{1,0.88,0.88}
\definecolor{LightGreen}{rgb}{0.88,1,0.88}

\definecolor{codegreen}{rgb}{0,0.6,0}
\definecolor{codegray}{rgb}{0.5,0.5,0.5}
\definecolor{codepurple}{rgb}{0.58,0,0.82}
\definecolor{backcolour}{rgb}{0.95,0.95,0.92}

\lstdefinestyle{mystyle}{
    backgroundcolor=\color{backcolour},   
    commentstyle=\color{codegreen},
    keywordstyle=\color{magenta},
    numberstyle=\tiny\color{codegray},
    stringstyle=\color{codepurple},
    basicstyle=\ttfamily\footnotesize,
    breakatwhitespace=false,         
    breaklines=true,                 
    captionpos=b,                    
    keepspaces=true,                 
    numbers=left,                    
    numbersep=5pt,                  
    showspaces=false,                
    showstringspaces=false,
    showtabs=false,                  
    tabsize=2
}

\lstset{style=mystyle}

\title{High-Performance Data Format for Scientific Data Storage and Analysis}

\author[1]{Gagik Gavalian}
\address[1]{Thomas Jefferson National Accelerator Facility, Newport News, VA, USA}

\begin{document}

%\begin{titlepage}
\begin{abstract}
In this article, we present the High-Performance Output (HiPO) data format developed at Jefferson Laboratory for 
storing and analyzing data from Nuclear Physics experiments. The format was designed to efficiently store large
amounts of experimental data, utilizing modern fast compression algorithms. The purpose of this development was 
to provide organized data in the output, facilitating access to relevant information within the large data files. The HiPO
data format has features that are suited for storing raw detector data, reconstruction data, and the final physics analysis 
data efficiently, eliminating the need to do data conversions through the lifecycle of experimental data. The HiPO data format
is implemented in C++ and JAVA, and provides bindings to FORTRAN, Python, and Julia, providing users with the choice of data 
analysis frameworks to use.
In this paper, we will present the general design and functionalities of the HiPO library and compare the performance of the library with
 more established data formats used in data analysis in High Energy and Nuclear Physics (such as ROOT and Parquete). In columnar data analysis, HiPO surpasses established data formats in performance and can be effectively applied to data analysis in other scientific fields.

\end{abstract}
%\end{titlepage}
\maketitle

\section{Introduction}

Over the decades, high-energy experiments, such as those conducted at particle accelerators like CERN’s Large Hadron Collider (LHC), have seen an exponential increase in data production. Early particle physics experiments in the mid-20th century relied on photographic plates and bubble chambers, which generated manageable amounts of data for manual analysis. As technology advanced, detectors became more sophisticated, capturing finer details of particle collisions across multiple dimensions, from spatial tracks to energy signatures. This evolution enabled scientists to probe deeper into the structure of matter and the fundamental forces of nature but came with an enormous uptick in data volume.

Modern experiments produce petabytes of data annually, thanks to the use of advanced digital detectors and high-frequency collision rates. For instance, the LHC can generate up to one billion proton-proton collisions per second during its operations. Each collision results in complex, high-dimensional data that must be recorded, processed, and analyzed. However, the vast majority of these collisions are routine and unremarkable, reflecting well-known physics processes. Only a tiny fraction contains the rare and novel events that could reveal new particles or phenomena, such as the discovery of the Higgs boson in 2012.

The challenge lies in efficiently processing this deluge of data to identify and retain only the relevant information while discarding the rest. This is achieved through a multi-layered system of data selection. First, hardware-based triggers operate in real-time to reduce data rates by orders of magnitude, selecting events based on basic characteristics like energy thresholds. Then, software-based algorithms provide more refined filtering by analyzing the remaining data for specific patterns or anomalies. Despite these strategies, the volume of "useful" data still reaches hundreds of terabytes, requiring vast computing resources and distributed networks to store and process the information.

Another key challenge is ensuring that the data reduction process does not inadvertently discard valuable signals. Designing selection algorithms requires balancing sensitivity to rare events with the need to filter out noise effectively. Advances in machine learning have become increasingly important in addressing this problem. Sophisticated models can analyze high-dimensional data for subtle correlations and anomalies, improving the ability to identify rare events. However, the computational demands of such models add another layer of complexity, requiring extensive computing power, energy, and expertise.

The rise of big data in high-energy physics not only highlights the field’s technological achievements but also underscores the growing need for innovative solutions in data management and analysis. Collaboration between physicists, computer scientists, and engineers will remain critical in addressing these challenges and unlocking the secrets hidden in the vast streams of experimental data.


\section{Motivation}
\label{section-motivation}

%Data stored from physics experiments consists of "events" that record information from a setup for one interaction of the incident beam particle
%with the target. These "events" are processed independently to identify particles and tracks identified by different detector components and construct a physics event, which is then used for high-level physics analysis. The information stored in one "event" is a collection of responses from all detector components with unique data structures. The data collected during the experiment undergoes several transformations before it reaches the stage where it is used for physics analysis.

Physics experiment data consists of "events," each representing information captured from a single interaction between an incident beam particle and the target. These events are processed individually to identify particles and reconstruct tracks using data from various detector components. This process builds a complete physics event, which serves as the foundation for high-level physics analysis. Each event encapsulates a collection of responses from all detector components, organized in unique data structures. Before reaching the stage of physics analysis, the collected data undergoes multiple transformations to prepare and refine it for further study.

\begin{itemize}
\item The data acquisition records data from all detector components for one instance of interaction in raw format  (usually times and accumulated charge in each component of each detector system) 
\item In the next stage, the raw data is transformed from its digital form to values with units, such as time in milliseconds and energy in eV.
\item The reconstruction program analyses data from each detector to identify related signals and combines signals from various detectors to identify particles in each event (collision instance). The produced output contains tables with information about the particles in the event and the responses of each particle in each detector component, helping to identify particle species.
\item For each physics analysis, different sets of selection algorithms are used to identify the physics reaction in each event and physics observables are calculated based on detected particles in the event, and the output is produced containing a columnar table for final physics analysis. 
\end{itemize}

In traditional CLAS~\cite{CLAS:2003umf} experiments, different data formats were used at each stage of the data lifecycle, leading to unnecessary complexity. This required supporting multiple file formats and maintaining numerous conversion tools. Additionally, users developed dozens of data selection and filtering tools tailored to specific formats, all of which required ongoing maintenance. Initially, a similar approach was considered for the CLAS12~\cite{Burkert:2020akg} experiment during its early software development stages.

However, experienced developers quickly recognized the challenges of this approach and envisioned a more streamlined solution. To address these issues, it was decided to adopt a single data format for all stages of the experimental data lifecycle. Several existing formats, such as ROOT~\cite{Brun:1997pa}, LCIO~\cite{Aplin:2012kj}, and HDF5~\cite{HDF5:2000pa}, were evaluated. While each had its strengths, none were found to efficiently support all stages of data transformation. Furthermore, the growing diversity of data analysis frameworks and programming languages introduced additional challenges, as seamless integration required appropriate language bindings—complicating the use of existing formats in unified workflows.

To overcome these limitations, the High-Performance Output (HiPO)~\cite{hipo5p0:2025jk} data format was developed specifically for CLAS12. HiPO is designed to efficiently handle all stages of experimental data processing, from reconstruction workflows to final columnar data analysis. It also provides language bindings for a variety of programming languages used within the collaboration, including C++, FORTRAN, Python, Java, and Julia, ensuring broad compatibility and streamlined workflows.


To ensure usability across all workflows of data processing, several key requirements were established for the data format:

%Several requirements were imposed to ensure usability in all workflows of data processing, as follows:
\begin{itemize}
\item {\bf Serializable:} The CLAS12 reconstruction workflow follows a Service-Oriented Architecture (SOA) that operates on a heterogeneous platform using message passing. The data format was designed to be easily serializable, enabling efficient transmission of event data to individual reconstruction services.
\item {\bf Compression Efficiency:} To minimize storage demands, the data format incorporates compression. The compression algorithm must balance speed and compression ratio, as high compression speed is critical for managing the large data volumes produced by experimental setups, particularly in high-rate nuclear physics experiments where maintaining high data throughput is essential.
\item {\bf Random Access Capability:} The format must support random access to specific data collections within a file. This functionality is vital for debugging, selectively writing subsets of data, and supporting multi-threaded applications that process data chunks asynchronously.
\item {\bf Data Grouping Functionality:} The format should enable the grouping of related datasets, allowing for efficient tagging or marking of different datasets. This capability facilitates the targeted reading of specific groups without requiring the processing of the entire dataset.
\end{itemize}

The reconstruction of experimental data in CLAS12 is written in Java, for this reason, Java is the primary development platform of the HiPO library, and the C++ library is developed in parallel, sometimes lagging in features, but they are being slowly ported to the C++ code. Most of the example codes in this article are Java, the equivalent C++ examples can be found in the repository.
There are experimental bindings to Python and Julia, which are not actively developed due to limited use by collaborators.
The subsequent chapters will explore the features of the HiPO data format in greater detail, with illustrative examples.

\section{Format Description}

Data from physics experiments is organized into small units called "events," with each event containing data related to a single physics interaction captured by the detector. These events are collected over a certain period and stored sequentially in a file. A HiPO file, designed for this purpose, is structured into the following components:

\begin{figure}[h!]
  \begin{center}
    \includegraphics[width=0.85\textwidth]{images/file_structure.pdf}
 \end{center}
  \caption{Schematic view of the HiPO file structure. The figure shows the conceptual design of the file structure. It's not an exact layout of the data structures of the headers. The detailed documentation is published on the GitHub repository with the source code.}
 \label{schema:file}
\end{figure}


\begin{itemize}
\item {\bf File Header:} This section includes essential information such as the file version, file length (used for consistency checks), the location of the file footer, and other relevant parameters.
\item {\bf User Header:} This part contains metadata that describes the content of the file. In dictionary-driven formats, it includes descriptors of the data stored within the file. Users can also add additional metadata when creating the file.
\item {\bf Data Records:} These are the compressed, grouped data from individual events. The size of data records is configurable at the time of file creation. Each record is assigned a user-defined identifier, known as a tag, to group similar events together.
\item {\bf File Footer:} The footer contains metadata about each data record in the file, including its position, size, and tag.
The general structure of a HiPO file is illustrated in Figure~\ref{schema:file}.
\end{itemize}

The schematic view of the file structure is shown in Figure~\ref{schema:file}.

\subsection{File Header}

The file header is a descriptor for the file containing information about the version of the file. This magic word identifies the file format and also carries information about the endianness of the data and the position of the file footer. The header also contains a block of data called "User Data". The user data includes dictionaries of the data objects stored in the file, as well as user-provided metadata describing the file or any parameters the creator may want to include with the file. This metadata comes in the form of key-value pairs usually provided by the user before the file initialization stage. To create a simple file with a given user header, use the code shown in Listing~\ref{lst:open_file}.
\rule{16.5cm}{0.4pt}
\begin{lstlisting}[language=java, caption=Java example for writing a file with meta-data., label=lst:open_file]
// open file with user-specified meta data
HipoWriter w = new HipoWriter(); // create the object
w.addConfig("date","created on 01/27/2021");
w.addConfig("author","John Pierce");
w.addConfig("type","Physics Events from A-Detector")
w.open("my_new_file.h5");
w.close();
\end{lstlisting}


\subsection{Records}

The records contain data on events stored in sequence. The record header contains information on the number of events stored, an array of indices pointing to each of the events in the buffer, and the length of the data payload before compression and after the compression. The maximum size of the records for each given file is configured at the creation of the file, and the default size is $8~MB$. The record header also contains a unique identifier, which is "0" by default, and events that are added to the file are recorded in sequence in the 
record until the record size limit is reached, at which point the record is compressed and persisted on the disk with the record header attached, and the record is cleared to start receiving the next events. This process continues until the file is closed, at which point the file footer is recorded, containing positions and identifiers of the records. The file header is updated with the position of the file footer for fast access when opening the file for reading.

\subsection{Events}

As it was mentioned above, the HiPO file consists of a series of events. An event is one unit containing a bunch of data objects that are related to each other. The types of data that can be stored in the event are arrays of primitive types and tables. Each object is assigned two identification numbers, which are used to 
retrieve these objects at the read. These identifiers are called "group" (16 bits) and "item" (8 bits). These logical identifiers can be used to group relevant data, use your imagination. The primitive types are simple arrays, and their headers describe the data structure unequivocally and they do not require dictionaries to be read. 

\subsubsection{Primitive types}

The primitive types are arrays of numbers and strings where all elements of the array have the same type (such as byte, short, integer, float, double, long, and string). The objects 
called Node are created from a provided array with user-defined identifiers. The example Listing~\ref{lst:write_arrays} shows how to create their primitive arrays and how to write them into an event:
\rule{16.5cm}{0.4pt}
\begin{lstlisting}[language=java, caption=Java example to create and write primitive types into an event, label=lst:write_arrays]
// Writing arrays into an Event
Event event = new Event(2048); // creta event with max size 2 kB
float[]  df = new float[]{1.0,2.0,3.0,4.0};
short[]  ds = new short[]{3,5,8,13,21,34,55};
Node   nf = new Node(12,1,df);
Node   ns = new Node(12,2,ds);
Node data = new Node(12,3,"Event recorded at 12:52:33"); 
event.write(nf);
event.write(ns);
event.write(data);
\end{lstlisting}

It's worth noting that the preliminary size of the event does not restrict the user to write objects that exceed the allocated event size. As new objects are added to the event, the event buffer 
will adopt the necessary size to accommodate the objects. It is recommended to set the size slightly larger than is needed to avoid reallocations for better performance.

\subsubsection{Tables}

Tables are rectangular data structures with columns and rows. Table objects require a schema (a descriptor) to be able to parse the content. The schema must be created 
and declared before the file is opened for writing since the file writer composes a dictionary and stores all schemas in the header of the file. The data structure that holds
tables is called a Bank and is also assigned two unique identifiers (group, item). The Listing~\ref{lst:write_bank} shows a simple example of how to declare and write a 
simple bank into a file:
\rule{16.5cm}{0.4pt}
\begin{lstlisting}[language=java, caption=Java example to create and write banks (tables) into an event, label=lst:write_bank]
// Writing banks to the event
SchemaBuilder b = new SchemaBuilder("data::clusters",12,1)
    .addEntry("type","B","cluster type") // B - type Byte
    .addEntry("n", "S", "cluster multiplicity") // S - Type Short
    .addEntry("x", "F", "x position") // F - type float
    .addEntry("y","F","y position") // F - type float
    .addEntry("z", "F", "z position"); // F - type Float
// -- create a cchema
  Schema schema = b.build();
// add schema to the file and open the file
  HipoWriter w = new HipoWriter();
  w.getSchemaFactory().addSchema(schema);
  w.addConfig("date","file created at 11:54:22 AM");
  w.addConfig("description","file contains clusters in calorimeter");
  w.open("clusters.h5");                
  Event event = new Event();
  Bank  b  = new Bank(schema,2); // create a table with 2 rows
  b.putByte("type", 0, (byte)   1); b.putByte("type", 1, (byte)   2);
  b.putShort("n"  , 0, (short) 13); b.putShort("n"  , 1, (short) 21);
  b.putFloat("x",0,0.1f); b.putFloat("x",1,0.2f);
  b.putFloat("y",0,1.1f); b.putFloat("y",1,1.2f);
  b.putFloat("z",0,2.1f); b.putFloat("z",1,2.2f);
  event.write(b);
  w.addEvent(event);
  w.close(); // close should be called to write the file footer
\end{lstlisting}

The methods to write and read values for each element in the bank have two interfaces, one using the name of the variable
which makes for more readable code, and the second method using the index of the column which provides better performance 
when needed. The same setters can be used as follows, b.putFloat("x",0,0.1f) $\rightarrow$ b.putFloat(2,0,0.1f), since "x" is the third 
column in the table (column indices are counted starting from "0"). The correct types of setters have to be used when writing the data 
to ensure that the values provided do not overflow the boundaries of the type. However, more generic getters can be used when reading 
the data, such as $getInt(entry, row)$ for all integer types and $getDouble(entry, row)$ for all floating point types.

\rule{15.5cm}{0.4pt}
\begin{lstlisting}[language=java, caption=Java example to read banks from the file, label=lst:read_bank]
HipoReader r = new HipoReader("clusters.h5");        
Bank[] banks = r.getBanks("data::clusters");
// to read more than on bank use: r.getBanks("a","b","c","d");
// if the bank is not present in the event, the returned object 
// will have getRows()==0, no error is generated.
while(r.nextEvent(banks)){
  System.out.printf("\%4d, \%5d, \%8.5f \%8.5f \%8.5f\n",
           banks[0].getInt("type", row),
           banks[0].getInt(1,row),
           banks[0].getFloat("x", row),
           banks[0].getFloat("y", row),
           banks[0].getFloat("z", row));
}
\end{lstlisting}

In Listing~\ref{lst:read_bank} shows how to read the banks from a file and print the content on the screen. The 
$getBanks(String... list)$ method accepts a list of the banks to be read in each event, so multiple banks can be
read and analyzed at once. More advanced examples of how to read events and query the content of the events 
can be found in the examples provided in the repository.

\begin{figure}[h!]
  \begin{center}
    \includegraphics[width=0.95\textwidth]{images/table_layouts.pdf}
 \end{center}
  \caption{Table types with their memory layout. Type 1 tables are optimized for copying and deleting rows from the table without much overhead on copying individual bytes, while Type 2 tables are better for compression, proving $7\%-15\%$ reduction in the bank size.}
 \label{fig:table_layouts}
\end{figure}

The HiPO provides two types of tables where the rows and columns are arranged differently. In the example above
each column from all the rows is grouped together into a contiguous memory, which makes it necessary to declare 
the number of rows at the bank creation time and the banks' size can not be changed on the fly. The second type
is when each row is one contiguous memory block and rows can be altered on the fly, which makes it easy to manipulate
rows programmatically by removing a row or copying a row into an equivalent table. The schematic view of memory
mapping for these two types of tables is shown in Figure~\ref{fig:table_layouts}. The reason for having two types of
tables is that one (the case with columns forming a contiguous memory) is better for data compression which makes
it more efficient for producing final data sets for analysis. The second table type is used for workflows where different
components work on the same data set that analyzes the tables by appending and removing entries from a given bank.
In the particular case of experimental physics usage is the data acquisition system, where a table is growing with incoming
data, and some rows are removed based on some conditions imposed by the analysis software. In our workflows, the table
where column values are grouped provides $7\%-15\%$ more compression. Examples of how to use different types of tables
can be found the the code repository.



%Example usages:
%\begin{verbatim}
%hipo::writer writer("output.file");
%for(int i = 0; i < 12000; i++){
%   hipo::event event = event_provider_next();
%    writer.addEvent(event);
%}
%writer.close();
%\end{verbatim}



%The data from physics experiments is stored in small units called "events", each event contains data related to one physics interaction from a detector. The file consists of a series of events accumulated by the experimental setup during a certain period of time. A HiPO file consists of the following parts:

%- File header: containing file version file length (for consistency check), file footer location, and some other relevant parameters.
%- User header: contains metadata describing the content of the file. In dictionary-driven content contains descriptors of the data 
%stored in the file, the user can add additional information at the creation of the file.
%- Data records: The actual event data grouped and compressed. The size of the data records is configurable at the file creation time. Each record is assigned a user-defined identifier (called tag), which is used to group similar events together.
%- File footer: Contains information about each record in the file, including position, size, and the tag of the record.
%A general structure of the HiPO file is shown in Figure~\ref{hipo_file_structure}.


\section{Event Tagging}

The HiPO is designed to store data in units of events, where events are collections of data related to each other (usually in time). In nuclear physics
experiments, these events refer to one interaction of a beam particle with the target and record particles produced by the interaction. After the data 
acquisition, the post-processing of the data identifies the number of particles produced and their properties and writes them in the event for further
physics analysis. Specific interactions occur with different frequencies in the physics experiment, and the analysis program analyzing specific physics 
reactions does not analyze every single event but rather requires a specific number of produced particles reconstructed to proceed with the analysis.

In Figure~\ref {fig:event_frequency}, a schematic view is shown of how data can be theoretically distributed in the file. Where event types, are defined by 
a number of rows in the table that lists the particles reconstructed by data processing software. It is evident that for analysis of rare interactions, most of
the data read is not useful and the program spends unnecessary I/O cycles putting pressure on a shared file system and computational resources.

\begin{figure}[h!]
  \begin{center}
    \includegraphics[width=0.85\textwidth]{images/tagged_records.pdf}
 \end{center}
  \caption{Schematic view of the file output for tagged and untagged events. When using the tagged mode of writing a file, events are organized in the records 
  based on the tag of the event, which is assigned by the user depending on the needs.}
 \label{fig:event_frequency}
\end{figure}

To solve this problem, the HiPO data format introduces a tagging feature for storing data. As mentioned before, each record kept in the file contains a unique
tag identifier and this information is stored in the file footer along with the record's positions and sizes. This allows sorting events into separate records during
the writing of the file, which will organize similar events together. The Listing~\ref{lst:write_tagged} shows an example of how to tag events by the number of rows contained in a table that lists clusters from the previous example. 

\rule{16.5cm}{0.4pt}
\begin{lstlisting}[language=java, caption=Java example to create and write primitive types into an event, label=lst:write_tagged]
// Writing arrays into an Event
HipoReader r = new HipoReader("myfile.h5");
// using HipoWriter.create() transfers all the dictionary
// and the metadata to the write object
HipoWriter w = HipoWriter.create("taggedfile.h5",r);

Bank[] b = r.getBanks("data::clusters");
Event event = new Event();
while(r.next(event)){
  event.read(b);
  if(b[0].getRows()>=5) event.setTag(5);
  if(b[0].getRows()==2) event.setTag(2);
  if(b[0].getRows()==3) event.setTag(3);
  w.addEvent(event);
}
w.close();
\end{lstlisting}

The example code with produce a file where all events with a matching number of rows are grouped together. the events with a specific number of
rows in the "clusters" bank can be read without any overhead of going through the entire file, and example of reading events where the number of
rows in the "clusters" bank are equal to 2 or are larger than 4 as shown in Listing~\ref{lst:read_tagged}.

\rule{16.5cm}{0.4pt}
\begin{lstlisting}[language=java, caption=Java example to sort events in the output file depending on number of rows in the table, label=lst:read_tagged]
// Writing arrays into an Event
HipoReader r = new HipoReader();
r.setTags(2,5); // read only tag=2 and tag=5
r.open("taggedfile.h5");
Bank[] b = r.getBanks("data::clusters");
while(r.nextEvent(b)){
   b[0].show();// print bank content on the screen
}
\end{lstlisting}

Only the bank "data::clusters" containing two rows and/or more than four rows is read in the Listing~\ref{lst:write_tagged}, however, then the event is written to the output file and the entire event with all the banks and nodes is written, but the decision on how to partition a file is based only on one bank. 
This feature is used in CLAS12 to tag events based on the event topology (number of detected particles and their types), which makes easier reading reading 
only relevant events from the data file for analysis.


%Example usages:
%\begin{verbatim}
%hipo::writer writer("output.file");
%for(int i = 0; i < 12000; i++){
%   hipo::event event = event_provider_next();
%    writer.addEvent(event);
%}
%writer.close();
%\end{verbatim}



%The data from physics experiments is stored in small units called "events", each event contains data related to one physics interaction from a detector. The file consists of a series of events accumulated by the experimental setup during a certain period of time. A HiPO file consists of the following parts:

%- File header: containing file version file length (for consistency check), file footer location, and some other relevant parameters.
%- User header: contains metadata describing the content of the file. In dictionary-driven content contains descriptors of the data 
%stored in the file, the user can add additional information at the creation of the file.
%- Data records: The actual event data grouped and compressed. The size of the data records is configurable at the file creation time. Each record is assigned a user-defined identifier (called tag), which is used to group similar events together.
%- File footer: Contains information about each record in the file, including position, size, and the tag of the record.
%A general structure of the HiPO file is shown in Figure~\ref{hipo_file_structure}.


\section{Columnar Data Storing}

The final stage of any data analysis is representing the data in a columnar format and fetching some subset of the data with constraints 
on other columns. At this stage, users operate on a large data set with many columns and rows, and access to all of the 
columns in the file is not required. The Apache Parquet~\cite{PARKQUET:2020jk} format is widely used in data science for storing columnar data, due to its tools
with Python data frames. It is highly tuned to store large data sets and provide fast access to required columns. The Parqute is gaining
some popularity in High Energy and Nuclear Physics with the growing popularity of Python as a physics analysis environment. In the
past two decades, the ROOT was the primary tool for physics analysis which also provides data structures for storing columnar data 
and accessing the columns selectively. It is worth mentioning that ROOT also has Python bindings. To extend the usage of HiPO files beyond
the data processing int physics analysis, and experimental implementation of columnar data storage were implemented and tested against the more established data formats.

\subsection{Design}

The flexibility of the HiPO data format allows storing data in columnar format, similar to Parquete and ROOT. 
The feature of assigning tags to records allows for writing the data in a columnar manner, where each column is assigned a tag and written separately into its own record.
This process is synchronized so that at every predetermined number of rows, every column data is serialized and outputted as a record. The schematic view of arranging 
the data in the file is shown in Figure~\ref{fig:tuple_schema}.

\begin{figure}[h!]
  \begin{center}
    \includegraphics[width=0.85\textwidth]{images/tuple_schema.pdf}
 \end{center}
  \caption{Schematic view of arranging each event in the file from columnar table.}
 \label{fig:tuple_schema}
\end{figure}

When reading the file, the desired columns (called branches) are declared, and only records (buckets) with corresponding tag numbers are read and deserialized.
The user program has access only to the declared branches. The files written as columnar data have to be read with the corresponding API to make sure that the columns 
are properly synchronized at the read time. Here is an example code to read a few columns from a file and fill in a histogram. 
%\begin{verbatim}
%\rule{16.5cm}{0.4pt}
\begin{lstlisting}[language=c++, caption=c++ example to read HiPO tuple file and fill a histogram., label=lst:read_tuple]
// open file and read-only specified branches
hipo::tuple tuple("tuple.h5", "c1:c2:c3:c4"); 
float data[4]; // declare a holder for the data to be read
twig::h1d h(120,-1.0,1.0); // declare a histogram
while(tuple.next(data)==true){
    h.fill(data[0]);
}
\end{lstlisting}
%\end{verbatim}

The example code shown on Listing~\ref{lst:read_tuple} opens a file to read branches names "c1"-"c4" from the file, reads all rows until reaching the end of the file passing them to the user code through a declared array. The values from the first branch ("c1") will be filled into a histogram.

\subsection{Benchmarking}

For reading tests, we produced a synthetic data set consisting of 24 columns and 50 M rows and created HiPO (4.8 GB), ROOT (4.4 GB), and Parquete (5.1 GB) files, all three with LZ4 compression. The columns were filled with randomly generated numbers in the range $0..1$. 

\begin{figure}[h!]
  \begin{center}
    \includegraphics[width=0.85\textwidth]{images/benchmark_final.pdf}
%     \includegraphics[width=0.45\textwidth]{images/data_frame_benchmark.pdf}
 \end{center}
  \caption{Reading benchmark for different file formats reading the columnar data with 24 columns and 50 M rows (file size $\approx 4.5 GB$). Only four columns are read, and histograms are filled for the respective columns. }
 \label{read_benchmark}
\end{figure}

Then, reading tests were performed by reading four branches (out of 24) from the data file and filling a histogram for each branch, representing a typical workflow for data exploration. The times were measured for ten consecutive reads, and the average time of the last four reads was used as the final result. The results are shown in Figure~\ref{read_benchmark}. As can be seen from the figure, the ROOT TTree has the worst performance however, the RNTuple, which is the new data format standard in the upcoming ROOT 7 release, provides significant improvements in data reading speeds, and its performance is similar to that of Paruet. 


\begin{table}[h!]
\centering
%\begin{tabular}{|l|c|c|c|c|} % '|' for borders, 'c' for center alignment
\begin{tabular}{|p{7cm}|p{3.0cm}|p{4.0cm}|}
\hline
\textbf{Format (Library)} & File Size (GB)  &  Execution time (sec)  \\ \hline \hline
Tuple HiPO Java loop      & 4.8&0.637 \\ \hline
Tuple HiPO Java DataFrame &4.8& 0.452 \\ \hline
Tuple HiPO C++  loop      &4.8& 0.675 \\ \hline
Tuple HiPO C++  DataFrame &4.8& 0.424 \\ \hline
Tuple HiPO Julia  DataFrame &4.8& 0.487 \\ \hline
Tuple Parquete DataFrame  &5.1& 1.812 \\ \hline
TTree ROOT C++ loop       &4.4 & 4.692 \\ \hline
TTree ROOT C++ DataFrame  &4.4 & 5.175 \\ \hline
RNTuple ROOT C++ loop     &3.8& 2.624 \\ \hline
RNTuple ROOT C++ loop (CINT)    &3.8&  5.283 \\ \hline
%--- 6.32 RNTuple ROOT C++ loop     &3.8& 1.843 \\ \hline
%--- 6.32 RNTuple ROOT C++ loop (CINT)    &3.8&  3.861 \\ \hline
Tuple HiPO Java loop (JShell)  &4.8&  0.672 \\ \hline
%ROOT TTree (C++)  &  4.4 GB & 106.40 & 5.37 sec&10.10 sec& 14.35 sec       \\ \hline
%HiPO Tuple   (C++)   & 4.8 GB & -  & 1.35 sec & 2.61 sec & 4.22 sec       \\ \hline
%HiPO Tuple  (Java)   & 4.8  GB & 5.82 & 2.16 sec & 3.78 sec & 5.84 sec     \\ \hline
%Parquete (Python)     &  5.1 GB & 15.97 & 1.80 sec & 3.56 sec & 5.44 sec        \\ \hline
%DataFrames (Julia)   &  4.8 GB & - & 0.48 sec & 0.70 sec & 0.96 sec      \\ \hline
%DataFrames  (Java)   & 4.8  GB & 5.82 & 0.35 sec & 0.53 sec& 0.57 sec   \\ \hline
%DataFrames  (C++)   & 4.8  GB & 5.82 & 0.55 sec & 0.73 sec& 0.94 sec   \\ \hline
\end{tabular}
\caption{Reading benchmark for different file formats reading four columns from a file and filling respective histograms for each column. The file is generated with 24 columns and 50M rows filled with random numbers [0..1]. }
\label{tab:read_benchmark}
\end{table}

The numerical values for these tests are summarized in Table~\ref{tab:read_benchmark}. 
The HiPO data format provides much faster reading speeds compared to Parquet and ROOT (RNtuple) in simple reading the data in the loop (as shown in Listing~\ref{lst:read_tuple}), both in C++ and Java implementations. The "DataFrame" in the tests indicates the experimental feature of HiPO where the histogram is passed to the tuple reader, and it uses bulk histogram fill from the data buckets, it provides $50\%$ improvement in filling the histograms.
The table also contains the preliminary benchmark in the Julia port of the HiPO library (not included in the graph), showing performance similar to compiled C++ code.

One thing to mention is that most of the exploratory data analysis is done in interactive shells using ROOT files where various combinations of selections can be used to plot selected variables. Our tests showed that the RNTuple performance drops significantly when using it in interactive mode (by a factor of 2), while using the HiPO Java library in interactive JShell does not affect the performance (seen in Table~\ref{tab:read_benchmark}). 
The columnar storage in HiPO is experimental, and the C++ source will be made public after intensive debugging. The Java code is available in the repository to run preliminary benchmarks.
%The numerical values for these tests are summarized in Table~\ref{tab:read_benchmark}. It is important to note that the writing performance of the data format is very important if it will be used in online for storing the experimental data. The faster serialization is preferable for the choice of data format. The writing times for the different formats are also summarized in Table~\ref{tab:read_benchmark}, and as it can be seen the writing times for HiPO are significantly shorter compared to both ROOT and Parquete.

%The benchmarks shown in Figure~\ref{read_benchmark} are performed using a loop over all elements in the tuple, where after each loading of the data, the control is returned to the user code where the histogram is filled (for ROOT and HiPO tuples). However, the Parquete (Python) code uses bulk fill when the histogram is filled from the data frame. This considerably speeds up the execution time due to reduced function calls. To have a fair comparison with the Parquet format in the HiPO library, bulk fill of histograms was implemented (both in Java, C++, and Julia), and the same tests were performed using bulk fill. The results of reading four columns and bulk-filling the values into histograms are shown in Figure~\ref{read_benchmark} (b). The comparison shows that HiPO is $4-5$ times faster than Parquet and about 10-15 times faster than reading four columns in ROOT.

The benchmarks are performed on an M1 Macbook Laptop with a 1 TB SSD drive, using JDK 21 for Java library and ROOT 6.34.08.

\section{Discussion}

The HiPO data format was developed for CLAS12 to serve as a data format for the entire life cycle of the experimental data. The CLAS12 reconstruction uses it for data processing and production data storage. Petabytes of data are stored in HiPO and have successfully been used by the collaboration to run their physics analysis workflows for the past 7 years. Since the introduction of the single data format, standard tools were developed for data file manipulations, such as data merging, filtering, and selective reduction, eliminating the need for users to write and maintain codes for the most common data manipulation tasks. There are tools (graphical and text-based) to browse the files and display the content of each event for debugging.
The recent development of columnar data storage can now also be used for storing columnar data from physics analysis. The tests show that it provides exceptional performance in data analysis, surpassing the more established data formats (ROOT and Parquet) currently used in the High Energy and Nuclear Physics fields.
Future improvements in binding to other languages, such as Python and Julia, will extend the usability of the data format for analysis.





\section{Acknowledgments}

This material is based upon work supported by the U.S. Department of Energy, Office of Science,
Office of Nuclear Physics under contract DE-AC05-06OR23177, and NSF grant no. CCF-1439079 and
the Richard T. Cheng Endowment. 
 
 
\begin{thebibliography}{}
%\cite{Aplin:2012kj}
%\cite{CLAS:2003umf}
\bibitem{CLAS:2003umf}
B.~A.~Mecking \textit{et al.} [CLAS],
``The CEBAF Large Acceptance Spectrometer (CLAS),''
Nucl. Instrum. Meth. A \textbf{503} (2003), 513-553
doi:10.1016/S0168-9002(03)01001-5
\bibitem{Burkert:2020akg}
Burkert, V.D. and others, The CLAS12 Spectrometer at Jefferson Laboratory, Nucl. Instrum. Meth. A \textbf{959},163419 (2020)
\bibitem{Brun:1997pa}
R.~Brun and F.~Rademakers,
``ROOT: An object oriented data analysis framework,''
Nucl. Instrum. Meth. A \textbf{389} (1997), 81-86
doi:10.1016/S0168-9002(97)00048-X

\bibitem{Aplin:2012kj}
S.~Aplin, J.~Engels, F.~Gaede, N.~A.~Graf, T.~Johnson and J.~McCormick,
``LCIO: A Persistency Framework and Event Data Model for HEP,''
doi:10.1109/NSSMIC.2012.6551478

\bibitem{HDF5:2000pa}
The HDF Group,  http://www.hdfgroup.org/HDF5/

\bibitem{PARKQUET:2020jk}
Apache Parquet, https://parquet.apache.org

\bibitem{hipo5p0:2025jk}
The HiPO Group, 
https://github.com/gavalian/hipo
\end{thebibliography}
%\newpage
%\bibliography{references}
%\bibliographystyle{ieeetr}

\end{document}
