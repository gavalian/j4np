\documentclass[preprint,12pt]{elsarticle}
\usepackage{graphicx}
\usepackage[margin=1.0in]{geometry}
\usepackage{color, colortbl}
\usepackage{hyperref}
\usepackage{float}
\usepackage{lineno}

%\linenumbers
% \usepackage[affil-it]{authblk}
\usepackage{subcaption}
\newcommand{\note}[1]{\textcolor{blue}{#1}}
\definecolor{LightCyan}{rgb}{0.88,1,1}
\definecolor{LightRose}{rgb}{1,0.88,0.88}
\definecolor{LightGreen}{rgb}{0.88,1,0.88}

\title{High-Performance Data Format for Scientific Data Analysis}

\author[1]{Gagik Gavalian}
\address[1]{Jefferson Lab, Newport News, VA, USA}



\begin{document}

%\begin{titlepage}
\begin{abstract}
In this article, we present the High-Performance Output (HiPO) data format developed at Jefferson Laboratory for 
storing and analyzing data from Nuclear Physics experiments. The format was designed to efficiently store large
amount of experimental data, utilizing modern fast compression algorithms. The purpose of this development was 
to provide organized data in the output, facilitating access to relevant information within the large data files. The HiPO
data format has features that are suited for storing raw detector data, reconstruction data, and the final physics analysis 
data efficiently, eliminating the need to do data conversions through the lifecycle of experimental data. The HiPO data format
is implemented in C++ and JAVA, and provides interfaces to Python and Julia, providing users with the choice of data 
analysis frameworks to use.
In this paper, we will present the general design and functionalities of the HiPO library and compare the performance of the library with
 more established data formats used in data analysis in High Energy and Nuclear Physics (such as ROOT and Parquete).

\end{abstract}
%\end{titlepage}
\maketitle


\section{Introduction}

Over the decades, high-energy experiments, such as those conducted at particle accelerators like CERN’s Large Hadron Collider (LHC), have seen an exponential increase in data production. Early particle physics experiments in the mid-20th century relied on photographic plates and bubble chambers, which generated manageable amounts of data for manual analysis. As technology advanced, detectors became more sophisticated, capturing finer details of particle collisions across multiple dimensions, from spatial tracks to energy signatures. This evolution enabled scientists to probe deeper into the structure of matter and the fundamental forces of nature but came with an enormous uptick in data volume.

Modern experiments produce petabytes of data annually, thanks to the use of advanced digital detectors and high-frequency collision rates. For instance, the LHC can generate up to one billion proton-proton collisions per second during its operations. Each collision results in complex, high-dimensional data that must be recorded, processed, and analyzed. However, the vast majority of these collisions are routine and unremarkable, reflecting well-known physics processes. Only a tiny fraction contains the rare and novel events that could reveal new particles or phenomena, such as the discovery of the Higgs boson in 2012.

The challenge lies in efficiently processing this deluge of data to identify and retain only the relevant information while discarding the rest. This is achieved through a multi-layered system of data selection. First, hardware-based triggers operate in real-time to reduce data rates by orders of magnitude, selecting events based on basic characteristics like energy thresholds. Then, software-based algorithms provide more refined filtering by analyzing the remaining data for specific patterns or anomalies. Despite these strategies, the volume of "useful" data still reaches hundreds of terabytes, requiring vast computing resources and distributed networks to store and process the information.

Another key challenge is ensuring that the data reduction process does not inadvertently discard valuable signals. Designing selection algorithms requires balancing sensitivity to rare events with the need to filter out noise effectively. Advances in machine learning have become increasingly important in addressing this problem. Sophisticated models can analyze high-dimensional data for subtle correlations and anomalies, improving the ability to identify rare events. However, the computational demands of such models add another layer of complexity, requiring extensive computing power, energy, and expertise.

The rise of big data in high-energy physics not only highlights the field’s technological achievements but also underscores the growing need for innovative solutions in data management and analysis. Collaboration between physicists, computer scientists, and engineers will remain critical in addressing these challenges and unlocking the secrets hidden in the vast streams of experimental data.


\section{Design}

In nuclear physics experiments, data goes through multiple stages of transformation before reaching the physics analysis phase. Initially, raw signals are collected from the experimental setup, capturing the direct measurements from each detector component. These raw signals are then processed to extract key features, such as pulse shapes and timing information. 

The subsequent step involves integrating the data from individual detector components. Signals are grouped and correlated across detectors to identify matches, enabling the reconstruction of particle trajectories and characteristics. Finally, the reconstructed particles are analyzed to determine their interactions with each detector component, providing the foundation for further physics analysis.

%Conventionally, different data formats are used in different stages of the data lifecycle, which presents several challenges,
%maintaining a few different libraries for reading and writing files and maintaining code for translating between the data formats.
%This is using CPU cycles for just transforming data structures. Also, different data persistence libraries are using different compression algorithms with varying throughput and efficiency. 

Traditionally, different stages of the data lifecycle employ distinct data formats, which introduces several challenges. Maintaining multiple libraries for reading and writing files, as well as developing and maintaining code for translating between these formats, can be resource-intensive. This process consumes valuable CPU cycles solely for transforming data structures.

The use of diverse data persistence libraries, each employing unique compression algorithms, creates inconsistencies in throughput and efficiency. These variations complicate optimization efforts and can negatively impact the overall performance of data processing workflows.

This project aimed to design a unified data format capable of efficiently handling data across all stages of experimental workflows while facilitating the analysis of large datasets. The requirements for the new format were informed by extensive experience with experimental data and include the following:
\begin{itemize}
\item {\bf Compression Efficiency:} The data format must incorporate compression to minimize storage requirements. The chosen compression algorithm should strike a balance between speed and compression ratio. High compression speed is critical due to the large data volumes generated by experimental setups, particularly in high-rate nuclear physics experiments where high data throughput is essential.
\item {\bf Random Access Capability:} The format must support random access to specific data collections within a file. This feature is crucial for debugging, enabling selective writing of data subsets, and for multi-threaded applications that process chunks of data asynchronously.
\item {\bf Data Grouping Functionality:} The format should allow for the grouping of related datasets. This capability is essential for marking or tagging different datasets, enabling targeted reading of specific groups without the need to process the entire dataset.
\end{itemize}
For a more detailed explanation of these features and examples of their implementation, please refer to the following text.

\section{Format Description}

Data from physics experiments is organized into small units called "events," with each event containing data related to a single physics interaction captured by the detector. These events are collected over a certain period and stored sequentially in a file. A HiPO file, designed for this purpose, is structured into the following components:

\begin{figure}[h!]
  \begin{center}
    \includegraphics[width=0.85\textwidth]{images/file_structure.pdf}
 \end{center}
  \caption{Schematic view of the HiPO file structure. The figure shows the conceptual design of the file structure. It's not an exact layout of the data structures of the headers. The detailed documentation is published on the GitHub repository with the source code.}
 \label{schema:file}
\end{figure}


\begin{itemize}
\item {\bf File Header:} This section includes essential information such as the file version, file length (used for consistency checks), the location of the file footer, and other relevant parameters.
\item {\bf User Header:} This part contains metadata that describes the content of the file. In dictionary-driven formats, it includes descriptors of the data stored within the file. Users can also add additional metadata when creating the file.
\item {\bf Data Records:} These are the compressed, grouped data from individual events. The size of data records is configurable at the time of file creation. Each record is assigned a user-defined identifier, known as a tag, to group similar events together.
\item {\bf File Footer:} The footer contains metadata about each data record in the file, including its position, size, and tag.
The general structure of a HiPO file is illustrated in Figure~\ref{schema:file}.
\end{itemize}

The schematic view of the file structure is shown in Figure~\ref{schema:file}.

\subsection{File Header}

The file header is a descriptor for the file containing information about the version of the file. This magic word identifies the file format and also carries information about the endianness of the data and the position of the file footer. The header also contains a block of data called "User Data". The user data includes dictionaries of the data objects stored in the file, as well as user-provided metadata describing the file or any parameters the creator may want to include with the file. This metadata comes in the form of key-value pairs usually provided by the user before the file initialization stage. To create a simple file with a given user header, use the code shown in Listing~\ref{lst:open_file}.
\rule{16.5cm}{0.4pt}
\begin{lstlisting}[language=java, caption=Java example for writing a file with meta-data., label=lst:open_file]
// open file with user-specified meta data
HipoWriter w = new HipoWriter(); // create the object
w.addConfig("date","created on 01/27/2021");
w.addConfig("author","John Pierce");
w.addConfig("type","Physics Events from A-Detector")
w.open("my_new_file.h5");
w.close();
\end{lstlisting}


\subsection{Records}

The records contain data on events stored in sequence. The record header contains information on the number of events stored, an array of indices pointing to each of the events in the buffer, and the length of the data payload before compression and after the compression. The maximum size of the records for each given file is configured at the creation of the file, and the default size is $8~MB$. The record header also contains a unique identifier, which is "0" by default, and events that are added to the file are recorded in sequence in the 
record until the record size limit is reached, at which point the record is compressed and persisted on the disk with the record header attached, and the record is cleared to start receiving the next events. This process continues until the file is closed, at which point the file footer is recorded, containing positions and identifiers of the records. The file header is updated with the position of the file footer for fast access when opening the file for reading.

\subsection{Events}

As it was mentioned above, the HiPO file consists of a series of events. An event is one unit containing a bunch of data objects that are related to each other. The types of data that can be stored in the event are arrays of primitive types and tables. Each object is assigned two identification numbers, which are used to 
retrieve these objects at the read. These identifiers are called "group" (16 bits) and "item" (8 bits). These logical identifiers can be used to group relevant data, use your imagination. The primitive types are simple arrays, and their headers describe the data structure unequivocally and they do not require dictionaries to be read. 

\subsubsection{Primitive types}

The primitive types are arrays of numbers and strings where all elements of the array have the same type (such as byte, short, integer, float, double, long, and string). The objects 
called Node are created from a provided array with user-defined identifiers. The example Listing~\ref{lst:write_arrays} shows how to create their primitive arrays and how to write them into an event:
\rule{16.5cm}{0.4pt}
\begin{lstlisting}[language=java, caption=Java example to create and write primitive types into an event, label=lst:write_arrays]
// Writing arrays into an Event
Event event = new Event(2048); // creta event with max size 2 kB
float[]  df = new float[]{1.0,2.0,3.0,4.0};
short[]  ds = new short[]{3,5,8,13,21,34,55};
Node   nf = new Node(12,1,df);
Node   ns = new Node(12,2,ds);
Node data = new Node(12,3,"Event recorded at 12:52:33"); 
event.write(nf);
event.write(ns);
event.write(data);
\end{lstlisting}

It's worth noting that the preliminary size of the event does not restrict the user to write objects that exceed the allocated event size. As new objects are added to the event, the event buffer 
will adopt the necessary size to accommodate the objects. It is recommended to set the size slightly larger than is needed to avoid reallocations for better performance.

\subsubsection{Tables}

Tables are rectangular data structures with columns and rows. Table objects require a schema (a descriptor) to be able to parse the content. The schema must be created 
and declared before the file is opened for writing since the file writer composes a dictionary and stores all schemas in the header of the file. The data structure that holds
tables is called a Bank and is also assigned two unique identifiers (group, item). The Listing~\ref{lst:write_bank} shows a simple example of how to declare and write a 
simple bank into a file:
\rule{16.5cm}{0.4pt}
\begin{lstlisting}[language=java, caption=Java example to create and write banks (tables) into an event, label=lst:write_bank]
// Writing banks to the event
SchemaBuilder b = new SchemaBuilder("data::clusters",12,1)
    .addEntry("type","B","cluster type") // B - type Byte
    .addEntry("n", "S", "cluster multiplicity") // S - Type Short
    .addEntry("x", "F", "x position") // F - type float
    .addEntry("y","F","y position") // F - type float
    .addEntry("z", "F", "z position"); // F - type Float
// -- create a cchema
  Schema schema = b.build();
// add schema to the file and open the file
  HipoWriter w = new HipoWriter();
  w.getSchemaFactory().addSchema(schema);
  w.addConfig("date","file created at 11:54:22 AM");
  w.addConfig("description","file contains clusters in calorimeter");
  w.open("clusters.h5");                
  Event event = new Event();
  Bank  b  = new Bank(schema,2); // create a table with 2 rows
  b.putByte("type", 0, (byte)   1); b.putByte("type", 1, (byte)   2);
  b.putShort("n"  , 0, (short) 13); b.putShort("n"  , 1, (short) 21);
  b.putFloat("x",0,0.1f); b.putFloat("x",1,0.2f);
  b.putFloat("y",0,1.1f); b.putFloat("y",1,1.2f);
  b.putFloat("z",0,2.1f); b.putFloat("z",1,2.2f);
  event.write(b);
  w.addEvent(event);
  w.close(); // close should be called to write the file footer
\end{lstlisting}

The methods to write and read values for each element in the bank have two interfaces, one using the name of the variable
which makes for more readable code, and the second method using the index of the column which provides better performance 
when needed. The same setters can be used as follows, b.putFloat("x",0,0.1f) $\rightarrow$ b.putFloat(2,0,0.1f), since "x" is the third 
column in the table (column indices are counted starting from "0"). The correct types of setters have to be used when writing the data 
to ensure that the values provided do not overflow the boundaries of the type. However, more generic getters can be used when reading 
the data, such as $getInt(entry, row)$ for all integer types and $getDouble(entry, row)$ for all floating point types.

\rule{15.5cm}{0.4pt}
\begin{lstlisting}[language=java, caption=Java example to read banks from the file, label=lst:read_bank]
HipoReader r = new HipoReader("clusters.h5");        
Bank[] banks = r.getBanks("data::clusters");
// to read more than on bank use: r.getBanks("a","b","c","d");
// if the bank is not present in the event, the returned object 
// will have getRows()==0, no error is generated.
while(r.nextEvent(banks)){
  System.out.printf("\%4d, \%5d, \%8.5f \%8.5f \%8.5f\n",
           banks[0].getInt("type", row),
           banks[0].getInt(1,row),
           banks[0].getFloat("x", row),
           banks[0].getFloat("y", row),
           banks[0].getFloat("z", row));
}
\end{lstlisting}

In Listing~\ref{lst:read_bank} shows how to read the banks from a file and print the content on the screen. The 
$getBanks(String... list)$ method accepts a list of the banks to be read in each event, so multiple banks can be
read and analyzed at once. More advanced examples of how to read events and query the content of the events 
can be found in the examples provided in the repository.

\begin{figure}[h!]
  \begin{center}
    \includegraphics[width=0.95\textwidth]{images/table_layouts.pdf}
 \end{center}
  \caption{Table types with their memory layout. Type 1 tables are optimized for copying and deleting rows from the table without much overhead on copying individual bytes, while Type 2 tables are better for compression, proving $7\%-15\%$ reduction in the bank size.}
 \label{fig:table_layouts}
\end{figure}

The HiPO provides two types of tables where the rows and columns are arranged differently. In the example above
each column from all the rows is grouped together into a contiguous memory, which makes it necessary to declare 
the number of rows at the bank creation time and the banks' size can not be changed on the fly. The second type
is when each row is one contiguous memory block and rows can be altered on the fly, which makes it easy to manipulate
rows programmatically by removing a row or copying a row into an equivalent table. The schematic view of memory
mapping for these two types of tables is shown in Figure~\ref{fig:table_layouts}. The reason for having two types of
tables is that one (the case with columns forming a contiguous memory) is better for data compression which makes
it more efficient for producing final data sets for analysis. The second table type is used for workflows where different
components work on the same data set that analyzes the tables by appending and removing entries from a given bank.
In the particular case of experimental physics usage is the data acquisition system, where a table is growing with incoming
data, and some rows are removed based on some conditions imposed by the analysis software. In our workflows, the table
where column values are grouped provides $7\%-15\%$ more compression. Examples of how to use different types of tables
can be found the the code repository.



%Example usages:
%\begin{verbatim}
%hipo::writer writer("output.file");
%for(int i = 0; i < 12000; i++){
%   hipo::event event = event_provider_next();
%    writer.addEvent(event);
%}
%writer.close();
%\end{verbatim}



%The data from physics experiments is stored in small units called "events", each event contains data related to one physics interaction from a detector. The file consists of a series of events accumulated by the experimental setup during a certain period of time. A HiPO file consists of the following parts:

%- File header: containing file version file length (for consistency check), file footer location, and some other relevant parameters.
%- User header: contains metadata describing the content of the file. In dictionary-driven content contains descriptors of the data 
%stored in the file, the user can add additional information at the creation of the file.
%- Data records: The actual event data grouped and compressed. The size of the data records is configurable at the file creation time. Each record is assigned a user-defined identifier (called tag), which is used to group similar events together.
%- File footer: Contains information about each record in the file, including position, size, and the tag of the record.
%A general structure of the HiPO file is shown in Figure~\ref{hipo_file_structure}.


\section{Discussion}

The HiPO data format was developed for CLAS12 to serve as a data format for the entire life cycle of the experimental data. The CLAS12 reconstruction uses it for data processing and production data storage. Petabytes of data are stored in HiPO and have successfully been used by the collaboration to run their physics analysis workflows for the past 7 years. Since the introduction of the single data format, standard tools were developed for data file manipulations, such as data merging, filtering, and selective reduction, eliminating the need for users to write and maintain codes for the most common data manipulation tasks. There are tools (graphical and text-based) to browse the files and display the content of each event for debugging.
The recent development of columnar data storage can now also be used for storing columnar data from physics analysis. The tests show that it provides exceptional performance in data analysis, surpassing the more established data formats (ROOT and Parquet) currently used in the High Energy and Nuclear Physics fields.
Future improvements in binding to other languages, such as Python and Julia, will extend the usability of the data format for analysis.





\newpage

\section{Acknowledgments}

This material is based upon work supported by the U.S. Department of Energy, Office of Science,
Office of Nuclear Physics under contract DE-AC05-06OR23177, and NSF grant no. CCF-1439079 and
the Richard T. Cheng Endowment. This work was performed using the Turing and Wahab computing
clusters at Old Dominion University.
 
\newpage
\bibliography{references}
\bibliographystyle{ieeetr}

\end{document}
