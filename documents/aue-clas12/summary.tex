

\section{Summary}

In this article we present a Machine Learning approach to de-noising detector data, the CLAS12 drift chambers specifically, using Convolutional Auto-Encoders. The data processed with the neural network and further processed with conventional tracking resulted in a significant increase in the number of reconstructed tracks. The study performed on simulated data shows a significant improvement of track reconstruction efficiency as a function of experimental luminosity. Using de-noising in combination with AI assisted tracking further improves the track reconstruction efficiency. The resulting increase in physics outcome was estimated to be $26\%$ for four particle final state in reaction $H(e,e^\prime\pi^+\pi^-)p$ for standard experimental luminosity for CLAS12 (at $45~nA$). The efficiency of track reconstruction at $120~nA$ is equal to the efficiency of conventional track reconstruction, which leads to the possibility of running experiments at a higher luminosity and accumulating the same physics statistics in $2.5$ times shorter time.
%Studies with experimental data confirm the increase in statistics of extracted physics observables and are in agreement with the simulation studies.
%Studies with higher luminosity show that by using de-noising and AI assisted tracking better track reconstruction efficiency can be achieved at twice higher luminosity. CLAS12 experiments can be carried out at higher luminosity without loss of statistical significance for the measured physics observables. 
%Given the increase in track reconstruction and significant increase in physics statistics for nominal data taking conditions at $45~nA$ we estimated savings of  $\$7.8$M USD annually. The calculation was done using publicly available budget for operating CEBAF~\cite{CEBAF:oper} adjusted for inflation~\cite{GoogleDotCom}. The calculated budget of ~\$$48$M for all four experimental halls annually means $\approx \$12$M operating budget for CLAS12 experiments.
%, and  $\approx 65\%$ increase in statistics leads to reported savings.