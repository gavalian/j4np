\section{Data Description}

\subsection{Monte-Carlo Simulation}

For these studies, we used physics reactions generated using Pythia Monte-Carlo~\cite{Pythia:2022} event generator, and generated events were processed with GEMC~\cite{gemc:2022} (GEANT4~\cite{geant4:2022} based detector simulation program) to produce data similar to experimental data. The four charged particle final state (namely $e^-,\pi^+,\pi^-,p$) is selected in the output of Pythia for our studies.
In addition, any number of neutral particles is allowed.

Simulated physics events were processed with the CLAS12 emulation software (GEMC) that produces raw signal data (similar to experimental).
Using these generated files new files were generated emulating different luminosity experimental conditions using CLAS12 standard background merging program \cite{Stepanyan:2020bg}. 

The background merging software uses real experimental data for given luminosity to extract background hits from all detector components that can later be overlayed on top of simulated data to emulate the realistic background conditions of the experiment. For our studies, we used background files from runs with beam currents of $45~nA$, $50~nA$, and $55~nA$. Combining them sequentially we generated data corresponding to $45~nA$, $95~nA$, and $150~nA$. The $95~nA$ data sample was produced by merging the $45~nA$ background file with the output of GEMC and then merging it with $50~nA$ background data. Similarly by merging $45~nA$, $50~nA$, and $55~nA$ in a sequence we obtained a data sample corresponding to $150~nA$. In further discussions, we refer to the original data sample simulated with Pythia and processed with GEMC without background as $0~nA$ data. All comparisons of single track efficiency and physics final state statistics 
are presented relative to those quantities obtained from the $0~nA$ data sample.

Most CLAS12 experiments so far have run with a $45~nA-50~nA$ beam on a liquid hydrogen target, and we want to measure the performance impact of the de-noising procedure for standard running conditions, and also see if we can run at higher beam currents (luminosity) which will increase the statistical power of experiments at given run time.

\subsection{Data Analysis}

To study the effect of the de-noising on particle reconstruction efficiency we processed the produced data samples through the stand-alone denoiser program to create de-noised counterparts of simulated data for each luminosity setting.  Both data samples were processed using the CLAS12 data reconstruction program. Then the track reconstruction efficiency was calculated for both data samples (original and de-noised) as a function of luminosity. The track reconstruction efficiency was calculated following the standard procedure for CLAS12~\cite{Stepanyan:2020bg}. The efficiency for positive tracks is defined as a ratio of 
events containing an electron and a positive hadron ($N_{eh^+}$) to the number of inclusive events with an electron reconstructed ($N_{e}$). The efficiency for negative tracks is calculated similarly:

\begin{equation}
L_t^+ = \frac{N_{h^+e}}{N_e} , L_t^- = \frac{N_{h^-e}}{N_e} 
\label{eq::eff}
\end{equation}
where $L_t^+$ is the multiplicity for positive particles and $L_t^-$ is the multiplicity for negatively
charged particles, respectively. In order to estimate the charged-particle reconstruction efficiency
as a function of the beam current, the multiplicity, $L_t^{+/-}$, is fitted with a linear function:
\begin{equation}
L_t^{+/-} = a + b\times I
\label{eq::eff2}
\end{equation}

Here $a$ and $c$ are the fit parameters and $I$ is the beam current. Then it is assumed that the
reconstruction efficiency, $E=1$ at $I=0$ nA:

\begin{equation}
E^{+/-} = 1 + c \times I
\label{eq::eff3}
\end{equation}
with $c=\frac{b}{a}$. The slope parameter $c$ represents the variation of the reconstruction
inefficiency per unit of the beam current ($nA$)~\cite{Stepanyan:2020bg}.

\subsection{Artificial Intelligence Assisted Tracking}

The CLAS12 data reconstruction software already contains neural networks helping to identify track candidates from combinations of clusters reconstructed in each of the super-layers of drift chambers~\cite{Gavalian:2020mlp}.
This network already provides a big improvement in the tracking efficiency compared to the conventional reconstruction algorithm. The impact on physics (depending on the number of particles in the reaction) is a $15\%-35\%$ increase in statistics. 
In the recently developed reconstruction software, the user can choose to use assistance from AI in identifying tracks or use purely the conventional algorithm to identify track candidates. In our studies, we first investigated the improvement of the de-noising algorithm by using the conventional algorithm to identify tracks. Then we extended these studies to include AI track identification when processing raw and de-noised data. By doing this we want to disentangle the performance improvements arising from de-noising and from AI assistance. 


 
