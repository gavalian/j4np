\documentclass[preprint,12pt]{elsarticle}
\usepackage{graphicx}
\usepackage[margin=1.0in]{geometry}
\usepackage{color, colortbl}
\usepackage{hyperref}
\usepackage{float}
% \usepackage[affil-it]{authblk}
\usepackage{subcaption}
\newcommand{\note}[1]{\textcolor{blue}{#1}}
\definecolor{LightCyan}{rgb}{0.88,1,1}
\definecolor{LightRose}{rgb}{1,0.88,0.88}
\definecolor{LightGreen}{rgb}{0.88,1,0.88}

\title{Covolutionoal Auto-Encoders for Drift Chamber data de-noising for CLAS12}
\author[1]{Gagik Gavalian}
\author[2]{Polykarpos Thomadakis}
\author[2]{Angelos Angelopoulos}
\author[2]{Nikos Chrisochoides}
\author[3]{Raffaella De Vita}
\author[1]{Veronique Ziegler}

% \affiliation[1]{organization={CRTC, Department of Computer Science, Old Dominion University}, city={Norfolk, VA}, country={USA}}
\address[1]{Jefferson Lab, Newport News, VA, USA}
\address[2]{CRTC, Department of Computer Science, Old Dominion University, Norfolk, VA, USA}
\address[3]{INFN, Sezione di Genova, 16146 Genova, Italy}
%Authored by Jefferson Science Associates, LLC under U.S. DOE Contract No. DE-AC05-06OR23177. The U.S. Government retains a non-exclusive, paid-up, irrevocable, world-wide license to publish or reproduce this manuscript for U.S. Government purposes.
%\fntext[fn1]{Authors contributed equally.}
%\fntext[fn2]{Correspoding author, \textit{gavalian@jlab.org}}

\begin{document}

%\begin{titlepage}
\begin{abstract}

  In this article we describe the implementation of Artificial Intelligence models in track reconstruction software for the CLAS12 detector at Jefferson Lab.
 The Artificial Intelligence based approach resulted in improved track reconstruction efficiency in high luminosity experimental conditions.  The track
 reconstruction efficiency increased by $15\%$ for single particle, and statistics in multi-particle physics reactions increased by $15\%-35\%$ depending 
 on the number of particles in the reaction. The implementation of artificial intelligence in the workflow also resulted in code speedup of $35\%$.
\end{abstract}
%\end{titlepage}
\maketitle


\section{Introduction}

Over the decades, high-energy experiments, such as those conducted at particle accelerators like CERN’s Large Hadron Collider (LHC), have seen an exponential increase in data production. Early particle physics experiments in the mid-20th century relied on photographic plates and bubble chambers, which generated manageable amounts of data for manual analysis. As technology advanced, detectors became more sophisticated, capturing finer details of particle collisions across multiple dimensions, from spatial tracks to energy signatures. This evolution enabled scientists to probe deeper into the structure of matter and the fundamental forces of nature but came with an enormous uptick in data volume.

Modern experiments produce petabytes of data annually, thanks to the use of advanced digital detectors and high-frequency collision rates. For instance, the LHC can generate up to one billion proton-proton collisions per second during its operations. Each collision results in complex, high-dimensional data that must be recorded, processed, and analyzed. However, the vast majority of these collisions are routine and unremarkable, reflecting well-known physics processes. Only a tiny fraction contains the rare and novel events that could reveal new particles or phenomena, such as the discovery of the Higgs boson in 2012.

The challenge lies in efficiently processing this deluge of data to identify and retain only the relevant information while discarding the rest. This is achieved through a multi-layered system of data selection. First, hardware-based triggers operate in real-time to reduce data rates by orders of magnitude, selecting events based on basic characteristics like energy thresholds. Then, software-based algorithms provide more refined filtering by analyzing the remaining data for specific patterns or anomalies. Despite these strategies, the volume of "useful" data still reaches hundreds of terabytes, requiring vast computing resources and distributed networks to store and process the information.

Another key challenge is ensuring that the data reduction process does not inadvertently discard valuable signals. Designing selection algorithms requires balancing sensitivity to rare events with the need to filter out noise effectively. Advances in machine learning have become increasingly important in addressing this problem. Sophisticated models can analyze high-dimensional data for subtle correlations and anomalies, improving the ability to identify rare events. However, the computational demands of such models add another layer of complexity, requiring extensive computing power, energy, and expertise.

The rise of big data in high-energy physics not only highlights the field’s technological achievements but also underscores the growing need for innovative solutions in data management and analysis. Collaboration between physicists, computer scientists, and engineers will remain critical in addressing these challenges and unlocking the secrets hidden in the vast streams of experimental data.



\input{datanalysis}

\input{datanalysis_ai}
%\section{Analysis of Track Reconstruction with AI}

After the implementation of track identification service in CLAS12 reconstruction software the outputs
from conventional tracking algorithm and artificial intelligence assisted tracking algorithm were analyzed
event by event to ascertain improvements of tracking. 
 
 \subsection{Particle Reconstruction efficiency}
 
 The Neural Network for track classification was trained on experimental data after it was processed with conventional tracking 
 reconstruction. Track that have "good" fit quality and were tracked back to the target were used as a training sample for both 
 MLP classifier and Auto-Encoder corruption fixing network. For detailer analysis of tracking reconstruction performance with and without 
 assistance from artificial intelligence we processed one run at nominal luminosity (45 nA) compare performances.
 
 %The efficiency of track reconstruction was obtained for separate track topologies (6 super-layer and 5 super-layer).
 \begin{figure}[!h]
\begin{center}
% \includegraphics[width=2.0in]{images/pos_theta_5SL.png}
  \includegraphics[width=5.0in]{images/summary_5SL_6SL_neg.pdf}
\caption { }
 \label{track:efficiency}
 \end{center}
\end{figure}

The results are shown on Figure~\ref{track:efficiency}, where dependence of number of reconstructed negatively charged 
 tracks are shown as a function of particle momentum (top row), polar angle in laboratory frame (bottom row) and interaction
 vertex (middle row). The reconstructed distributions from conventional tracking are plotted with filled histograms and the
 tracks reconstructed by using assistance from AI are plotted with solid lines. As can be seen from the figure there is very big gain 
 in number of reconstructed tracks with 5 super-layer configuration compared to full 6 super-layer tracks. Typically for nominal 
 45 nA experimental data increase track efficiency for 6 super-layers tracks averages about $3\%-6\%$, while for 5 super-layer
 tracks the increase is in the order of $70\%-120\%$. In normal data reconstruction tracks that are identified with 5 super-layers
 usually comprise about $10\%$ of all reconstructed tracks, and significant increase in identification of such tracks leads to 
 overall tracking efficiency increase of $12\%-15\%$. 
 
 \begin{table}[!h]
 \begin{center}
 \begin{tabular}{|l|c|c|c|c|}
 \hline
 Track Configuration & Conventional & AI Assisted & Gain & Relative \\
 \hline
 \hline
 6 Super-Layer & 242,145 & 256,175 & 14030 & 1.0579 \\
 5 Super-Layer & 24,155 & 52,839 & 28684 & 2.1875 \\
 All & 267,339 & 309,058 & 51719 & 1.1561 \\
 \hline
 \end{tabular}
 \end{center}
 \caption{Summary of reconstructed tracks and gain with assistance from Artificial Intelligence.}
 \label{tbl:summary}
 \end{table}
 
The comparison of 5 super-layer and 6 super-layer track statistics and their relative gain is summarized in Table~\ref{tbl:summary}.
As can be seen from the table the gain in only 6 super-layer tracks is about $5.7\%$ but with significant gain in 5 super-layer tracks 
the overall gain in reconstructed tracks elevates to $>15\%$. These results are intuitive since track candidates composed of 5
super-layers with the same number of clusters in each super-layer are significantly higher than 6 super-layer track candidates, and 
it our tests AI performs better in choosing right combination with increasing combinatorics.
 
\subsection{Luminosity Dependence}

Track reconstruction efficiency increased with AI assisted tracking, since AI can better identify tracks
from pool of candidates. One would expect that if the number of combinations decrease efficiency 
of conventional track selection algorithm should approach efficiency of AI assisted track identification,
similarly, when number of combination increases the advantage of AI over the conventional algorithm should
increase. Based on this we expect AI to perform better in higher background settings. To evaluate AI assisted
tracking efficiency dependence on background we analyzed several different runs that were taken in different 
conditions (i.e. beam current) ranging from $5~nA$ to $70~nA$. To measure tracking efficiency we first calculated
number of electrons ($N_e$) detected in the data sample analyzed (typically one run) and then number of positive and negative
 hadrons that were detected with the electron inclusively ($N_{h^+e}$ and $N_{h^-e}$ respectively).

Then the efficiency for the data set was calculated as:

\begin{equation}
L_t^+ = \frac{N_{h^+e}}{N_e} , L_t^- = \frac{N_{h^-e}}{N_e} 
\end{equation}

where $L_t^+$ is the efficiency of positive particles and $L_t^-$ is the efficiency of negatively charged particles respectively. 
In order to estimate the charged particle reconstruction efficiency as a function of the beam current, the multiplicity, $L_t^{+/-}$, is fitted with a linear function:
\begin{equation}
L_t^{+/-} = a + c\times I 
\end{equation}

Here $a$ and $c$ are the fit parameters and $I$ is the beam current. Then it was assumed that the reconstruction efficiency, $E=1$ at $I=0$ nA:

\begin{equation}
E^{+/-} = 1 + b \times I 
\end{equation}

with $b=\frac{c}{a}$. The slope parameter b is the rate of the reconstruction inefficiency as a function of the beam current \cite{Stepanyan:2020bg}.
%The all points were 
%fitted with linear function $L=a+bx$, where $a$ is the intercept 
 
 \begin{figure}[!ht]
\begin{center}
 \includegraphics[width=3.0in]{images/lumi_scan_positive.pdf}
 \includegraphics[width=3.0in]{images/lumi_scan_negative.pdf}
\caption {Tracking efficiency for positively and negatively charged particles as a function of beam current (luminosity).  Conventional algorithm 
track reconstruction efficiency (diamonds) is compared to AI assisted track reconstruction efficiency (circles). }
 \label{lumi:scan}
 \end{center}
\end{figure}

The comparison of tracking efficiency as a function of beam current (luminosity) can be seen on Figure~\ref{lumi:scan} where $E^{+/-}$ are shown for positively and negatively charged particles separately. As can be seen from the figure AI assisted tracking performs significantly better for any given luminosity (beam current) and the decrease of efficiency is much slower as function of luminosity, $0.22\%$ per nA versus $0.40\%$ per nA for conventional tracking. This is expected and consistent with assumption that with increased combinatorial background (increased number of track candidates to consider) , AI performs better in choosing best track candidate. We established that AI assisted
tracking leads to more tracks reconstructed for any given beam current setting, next thing to check is what is the impact of increase track reconstruction
efficiency on physics analysis, and if there is increase in physics outcome for the CLAS12 experimental setup.

\subsection{Physics Impact}

To measure practical implications of track reconstruction efficiency improvement on physics outcome we analyzed 
two event topologies with two particle and three particles in the final state respectively. The data for analysis were 
taken with $10.5~GeV$ electron beam incident on $20~cm$ liquid hydrogen target, with the beam current of $45~nA$
(typical for CLAS12 experimental running). We selected events where an electron was detected in the forward detector 
then isolated events where there was an additional negatively charge pion ($\pi^-$) along with electron and no other 
charged particle, and the second topology required two pions along with electron, one positively charged and one 
negatively charged. The two chosen topologies are denoted by $H(e,e'\pi^-X)$ and $H(e,e'\pi^+\pi^-X)$. In both cases 
there is a visible peak of a missing proton that we can use to measure impact of efficiency on physics outcome. 

 \begin{figure}[!ht]
\begin{center}
 \includegraphics[width=6.0in]{images/physics_results.pdf}
\caption {Architecture of corruption fixing Auto-Encoder.}
 \label{physics:outcome}
 \end{center}
\end{figure}

The distributions of missing mass for both final state topologies are shown on Figure~\ref{physics:outcome}, the plots 
on the top row are missing mass of $H(e,e'\pi^-X)$ and $H(e,e'\pi^+\pi^-X)$, where the filled histogram is calculated from 
particles reconstructed by conventional tracking algorithm, and the histogram with black outline are same distributions 
calculated from particles that were reconstructed using suggestion from Artificial Intelligence. As can be seen from the figure 
there is significant increase in number of events under the missing proton peak (at mass value $0.938~MeV$) for AI assisted
tracking. The ratios of two histograms (AI assisted divided by Conventional) can be seen on the bottom row of 
Figure~\ref{physics:outcome}. As can be seen from the figure the increase in statistics is uniform over the whole range of the 
missing mass indicating no systematic abnormalities for AI assisted tracking. The ratio also indicates that there is an increase 
for number of event under the peak for proton, about $15\%$ for $H(e,e'\pi^-X)$ final state and $30-35\%$ for the $H(e,e'\pi^+\pi^-X)$
final state. Further studies show that increase in statistics for different final states increase with increase of beam current (luminosity) 
which is consistent with our studies of increased efficiency of single particle reconstruction. 


\section{Summary}

In this paper we investigated results of analysis of experimental data from CLAS12 detector reconstructed with assistance of Artificial Intelligence
to identify tracks from the hits in Drift Chambers. This work is based on two Neural Networks developed to classify track candidates from
given cluster combinations \cite{Gavalian:2020oxg} and to identify missing cluster positions in tracks that do not have complete 6 cluster configuration \cite{Gavalian:2020xmc}. After implementing these networks into CLAS12 reconstruction workflow, the AI was able to identify "good" track candidates 
and pass them to tracking code to be analyzed parallel to conventional algorithm that chooses "good" track candidates iteratively considering all possible combinations. 
Our studies showed that AI assisted tracking performs better than conventional track identification algorithm, and leads to track reconstruction efficiency increase of $15\%$ for nominal experimental running conditions (beam current 45 nA). The AI also performs better with increasing background (i.e. with increased incident beam current) and improves the efficiency loss from $0.44\%$ per nA to $0.24\%$ per nA.
This increased track reconstruction (identification) efficiency directly impacts the outcome of physics analysis where it increases statistics 
for physics reactions for $15\%-35\%$ depending on how many particles are detected in the final state and the topology of the reaction. This has big implication on experimental running conditions, since with increased efficiency required statistical significance of the experiment can be reached in shorter time by running at higher beam current (luminosity). Already collected experimental data be re-processed with the AI assisted tracking
code which can increase the statistics for analyzed data up to $35\%$. Both, future experiments and already completed experiment will benefit 
from this novel development.

Another important outcome of this development was reduction in data processing times, since track candidates were identified by Artificial Intelligence 
there were fewer marginal quality tracks picked to be analyzed and then later dropped due to non convergence of Kalman-Filter, and this leads to tracking code speed-up of $35\%$.

Overall we identified that Artificial Intelligence approach to assisting tracking codes is a good approach, it leads to improvements in code speed and 
efficiency of track reconstruction. Another important aspect of using Artificial Intelligence is that is leads to very small and simple code base, comprised of composing track candidates and feeding them to neural network, and what's also important it keeps improving with constant training with new data.
We intend to continue this development in extending the approach to other tracking detectors of CLAS12, and possibly try to adopt  our approach for other experimental detector setups at Jefferson Lab.



\newpage

\section{Acknowledgments}

This material is based upon work supported by the U.S. Department of Energy, Office of Science,
Office of Nuclear Physics under contract DE-AC05-06OR23177, and NSF grant no. CCF-1439079 and
the Richard T. Cheng Endowment. The authors would like to thank Raffaella De Vita for help in
processing data with CLAS12 reconstruction software. This work was performed using the Turing
and Wahab computing clusters at Old Dominion University.
 
\newpage
\bibliography{references}
\bibliographystyle{ieeetr}

\end{document}
