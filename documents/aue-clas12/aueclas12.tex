\documentclass[preprint,12pt]{elsarticle}
\usepackage{graphicx}
\usepackage[margin=1.0in]{geometry}
\usepackage{color, colortbl}
\usepackage{hyperref}
\usepackage{float}
\usepackage{lineno}

%\linenumbers
% \usepackage[affil-it]{authblk}
\usepackage{subcaption}
\newcommand{\note}[1]{\textcolor{blue}{#1}}
\definecolor{LightCyan}{rgb}{0.88,1,1}
\definecolor{LightRose}{rgb}{1,0.88,0.88}
\definecolor{LightGreen}{rgb}{0.88,1,0.88}

\title{Convolutional Auto-Encoders for Drift Chamber de-noising for CLAS12}
\author[1]{Gagik Gavalian}
\author[2]{Polykarpos Thomadakis}
\author[2]{Angelos Angelopoulos}
\author[2]{Nikos Chrisochoides}

\address[1]{Jefferson Lab, Newport News, VA, USA}
\address[2]{CRTC, Department of Computer Science, Old Dominion University, Norfolk, VA, USA}


\begin{document}

%\begin{titlepage}
\begin{abstract}
In this article, we present the results of using Convolutional Auto-Encoders for de-noising raw data for CLAS12 drift chambers.
The de-noising neural network provides increased efficiency in track reconstruction, also improved performance for high 
luminosity experimental data collection. The de-noising neural network used in conjunction with the previously developed track 
classifier neural network~\cite{Gavalian:2022hfa} lead to a significant track reconstruction efficiency increase for current luminosity
($0.6\times10^{35}~cm^{-2}~sec^{-1}$ ). The increase in experimentally measured quantities will allow running experiments at twice 
the luminosity with the same track reconstruction efficiency. This will lead to huge savings in accelerator operational costs, and large 
savings for Jefferson Lab and collaborating institutions.
\end{abstract}
%\end{titlepage}
\maketitle


\section{Introduction}

Over the decades, high-energy experiments, such as those conducted at particle accelerators like CERN’s Large Hadron Collider (LHC), have seen an exponential increase in data production. Early particle physics experiments in the mid-20th century relied on photographic plates and bubble chambers, which generated manageable amounts of data for manual analysis. As technology advanced, detectors became more sophisticated, capturing finer details of particle collisions across multiple dimensions, from spatial tracks to energy signatures. This evolution enabled scientists to probe deeper into the structure of matter and the fundamental forces of nature but came with an enormous uptick in data volume.

Modern experiments produce petabytes of data annually, thanks to the use of advanced digital detectors and high-frequency collision rates. For instance, the LHC can generate up to one billion proton-proton collisions per second during its operations. Each collision results in complex, high-dimensional data that must be recorded, processed, and analyzed. However, the vast majority of these collisions are routine and unremarkable, reflecting well-known physics processes. Only a tiny fraction contains the rare and novel events that could reveal new particles or phenomena, such as the discovery of the Higgs boson in 2012.

The challenge lies in efficiently processing this deluge of data to identify and retain only the relevant information while discarding the rest. This is achieved through a multi-layered system of data selection. First, hardware-based triggers operate in real-time to reduce data rates by orders of magnitude, selecting events based on basic characteristics like energy thresholds. Then, software-based algorithms provide more refined filtering by analyzing the remaining data for specific patterns or anomalies. Despite these strategies, the volume of "useful" data still reaches hundreds of terabytes, requiring vast computing resources and distributed networks to store and process the information.

Another key challenge is ensuring that the data reduction process does not inadvertently discard valuable signals. Designing selection algorithms requires balancing sensitivity to rare events with the need to filter out noise effectively. Advances in machine learning have become increasingly important in addressing this problem. Sophisticated models can analyze high-dimensional data for subtle correlations and anomalies, improving the ability to identify rare events. However, the computational demands of such models add another layer of complexity, requiring extensive computing power, energy, and expertise.

The rise of big data in high-energy physics not only highlights the field’s technological achievements but also underscores the growing need for innovative solutions in data management and analysis. Collaboration between physicists, computer scientists, and engineers will remain critical in addressing these challenges and unlocking the secrets hidden in the vast streams of experimental data.



\section{CLAS12 Drift Chambers}

The Drift Chambers (DC), which are  part of the large detector system of CLAS12 located in the experimental 
Hall-B at Jefferson Lab. They are used for charged particle detection in the forward direction 
(covering polar angles $5-35^\circ$). The CLAS12 forward detector is built around a six-coil 
toroidal magnet which divides the active detection area into six azimuthal regions, called “sectors”. 
For each sector, there are separate drift chambers installed consisting of 3 regions. Each region contains 
two super-layers, each of them containing 6 layers of wires.   Each layer of the drift chamber 
consists of 112 signal wires making each region a matrix of 12x112. The raw signal from 
one sector makes a matrix of 36x112, which is analyzed independently from other sectors
to extract trajectories of charged particles from raw signals. 

 \begin{figure}[!h]
\begin{center}
 \includegraphics[width=3in]{images/dc_region_2_with_noise.pdf}
 \includegraphics[width=3in]{images/dc_region_2_no_noise.pdf}
\caption {Example of clustering for one region of Drift Chambers. The left panel shows 
all the hits detected in the drift chamber (for this particular region), and the right panel 
shows results of clustering where some hits were identified as a background and were removed,
and the remaining hits were grouped to form a cluster.}
 \label{conv:denoising}
 \end{center}
\end{figure}

Each super-layer is analyzed separately for each sector and hits grouped together along the track trajectory 
are combined into clusters (or segments). In Figure~\ref{conv:denoising} the procedure is shown for one
region where all the hits (dark gray) are shown on the left panel, and clusters (red) are shown on the right panel,
by grouping neighboring wires after removing noise hits. Each super-layer
may have multiple clusters. The tracking algorithm creates a list of track candidates consisting of one cluster 
per super-layer and then analyzes the list to determine which candidates form a valid track. The identified 
tracks are further refined by passing them through Kalman filter~\cite{Kalman1960}. 
Examples of analyzed events in one sector can be seen in Figure~\ref{conv:trackfinding}, where 36x112 matrices
for four sectors are shown (not from the same event) with all signal hits in all layers (top row). The hits 
for clusters for identified tracks are shown on the bottom row. 


 \begin{figure}[!h]
\begin{center}
 \includegraphics[width=6.0in]{images/dc_example_all_hits.pdf}
 \includegraphics[width=6.0in]{images/dc_example_track_hits.pdf}
\caption {Example of reconstructed tracks in drift chambers. The signal hits in drift chambers are shown 
on the top row. The hits (clusters) belonging to identified tracks are shown on the bottom row. Dashed 
lines represent the boundaries of super-layers.}
 \label{conv:trackfinding}
 \end{center}
\end{figure}

As can be seen from the figure one or multiple tracks can be detected in one sector for the event. The efficiency
of finding these tracks depends on the cluster finding algorithm. With increased luminosity, the number of background 
hits increases, and it becomes difficult to separate background hits from signal hits due to heavy overlap between them.
This results in lost clusters and eventually in a decrease in track finding efficiency. 
In this work, Machine Learning is used to remove background hits prior to the clustering algorithm to improve cluster 
finding and consequentially track finding efficiency. The reconstructed experimental data is used to train Convolutional
Auto-Encoder for de-noising the drift chamber signal~\cite{Thomadakis:2022zcd}.



\section{Neural Network}

The Convolutional Auto-Encoder was used to de-noise raw data from CLAS12 drift chambers~\cite{Thomadakis:2022zcd}. The input and output for the network are matrices of size 36x112. The training data was extracted from experimental data. The raw hits (converted into a matrix) were used as an input for the neural network and a matrix constructed only from hits that belong to reconstructed tracks an output. In training data set multiple track hits were allowed in the output matrix. The structure of neural network can be seen on Figure~\ref{network:cnn_encoder}.

\begin{figure}[!h]
\begin{center}
 \includegraphics[width=5.1in]{images/convolutional-autoencoder.png}
\caption {De-noising neural network architecture.}
 \label{network:cnn_encoder}
 \end{center}
\end{figure}

The network was validated on experimental data where number of hits along the track trajectory were compared for de-noised data.
Example of comparison can be seen in Figure~\ref{network:cnn_results} where raw data (left column) is shown along with data with hits
belonging to reconstructed track (middle column) and reduced data passing through de-nosing program (right column).

\begin{figure}[!h]
\begin{center}
 \includegraphics[width=5.8in]{images/cnn_denoise_results.pdf}
\caption {De-noising neural network architecture.}
 \label{network:cnn_results}
 \end{center}
\end{figure}

As can be seen from the figure de-noiser removes all background hits not associated with a track. Systematic studies~\cite{Thomadakis:2022zcd} showed that more than $95\%$ of the track related hits are preserved in the output of de-noiser while background hits are significantly suppressed.

For out implementation we used TensorFlow/Keras~\cite{keras-website} to train and evaluate the network. The resulting network parameters (weights) were saved in HDF5 file. The de-noiser implementation for CLAS12 reconstruction software is done using DeepLearning4J~\cite{dl4j-website} which supports model imports through HDF5 files. The de-noising is not yet implemented as a part of CLAS12 reconstruction workflow, and works as a standalone package to process raw data before analyzing with reconstruction software.


\section{Data Description}

\subsection{Monte-Carlo Simulation}

For these studies we used physics reactions generated using Pythia event generator and selected events were processed with GEMC (GEANT based detector simulation program) to produce data similar to experimental data. From generic Pythia output only events that contain four charged particle in as decay products in the final state (namely $e^-,\pi^+,\pi^-,p$) and any number of neutral particles. These events were used as an input to GEMC program that emulates CLAS12 detector responses and produces files with raw detector information.

Using these generated files new files were generated emulating different luminosity experimental conditions using CLAS12 standard background merging program \cite{Stepanyan:2020bg}. The background merging software uses real experimental data for given luminosity to extract background hits from all detector components that can later be overlayed on top of generated data to emulate high background running conditions. For our studies we used background files corresponding to $45~nA$, $50~nA$ and $55~nA$. Combining them sequentially we generated data corresponding to $45~nA$, $95~nA$ and $150~nA$. The $95~nA$ data sample was produced by merging $45~nA$ background file with the output of GEMC and then merging it with $50~nA$ background data. Similarly by merging $45~nA$, $50~nA$ and $55~nA$ in sequence we obtained data sample corresponding to $150~nA$. 

Most CLAS12 experiments run with $45~nA$ electron beam, and we want to measure performance impact of de-noising procedure for standard running conditions, and also see if we can run at higher beam currents (luminosity) which will potentially increase the statistical power of experiments.

\subsection{Data Analysis}

To study the effect of de-noising on particle reconstruction efficiency we processed produced data samples through stand-alone de-noiser program to produce de-noised counterparts of simulated data for each luminosity setting. The both data samples were processed using CLAS12 data reconstruction program. Then the track reconstruction efficiency was calculated for both data samples (original and de-noised) as a function of luminosity. The track reconstruction efficiency was calculated following standard (for CLAS12) procedure~\cite{Stepanyan:2020bg}. The efficiency for positive  tracks is defines as a ratio of 
events containing an electron and a positive  hadron ($N_{eh^+}$) to the number of inclusive events with an electron reconstructed ($N_{e}$). The efficiency of negative tracks is calculated similarly:

\begin{equation}
L_t^+ = \frac{N_{h^+e}}{N_e} , L_t^- = \frac{N_{h^-e}}{N_e} 
\label{eq::eff}
\end{equation}

Track reconstruction efficiencies were compared for regular and de-noised data samples for different luminosity settings.

\subsection{Artificial Intelligence Assisted Tracking}

The CLAS12 data reconstruction software already contains neural networks helping to identify track candidates from combinations of clusters reconstructed in each of the super-layers of drift chambers~\cite{Gavalian:2022mlp}.
This network already provides big improvement of tracking efficiency compared to traditional reconstruction algorithm. The impact on physics (depending on number of particles in the reaction) is $15\%-35\%$ increase in statistics. 
In the standard reconstruction software user can chose to use assistance from AI in identifying tracks or use purely the conventional algorithm to identify track candidates. In our studies we first investigated the improvement of de-noising algorithm by using the conventional algorithm to identify tracks. Then we extended these studies to include AI track identification when processing raw and de-noised data. By doing this we want to disentangle the performance improvements arising from de-noising from AI assistance. 


 


\section{Data Analysis with De-Noising}

In this section we compare results from analysis of raw background merged data sample with files that were de-noised prior to running through CLAS12 reconstruction software. The comparison was done for files with different background merged (namely $45~nA$, $95~nA$ and $150~nA$). The data for raw sample and de-nosied sample was processed with same settings of CLAS12 reconstruction software.

\subsection{Luminosity dependence}

The track reconstruction efficiency was calculated according to Eq.~\ref{eq::eff} for positive and negative charged particles. The results are shown on Figure~\ref{lscan::conv_dn}. 

\begin{figure}[!h]
\begin{center}
 \includegraphics[width=3.1in]{images/figure_lscan_pos.pdf}
 \includegraphics[width=3in]{images/figure_lscan_neg.pdf}
\caption {Tracking efficiency as a function of luminosity (beam current) for positive (a) and negative particle (b).  The efficiency is shown for
conventional algorithm running on background merged files (diamonds), and on files with merged background then de-noised with AI (circles).}
 \label{lscan::conv_dn}
 \end{center}
\end{figure}

As can be seen from the figure the number of reconstructed hadron-electron pairs relative to reconstructed electrons is higher for de-noised data sample compared to the raw data sample. This is due to increased number of clusters reconstructed by conventional clustering algorithm in de-noised data samples. Detailed studies of cluster reconstruction efficiency were performed in our studies of neural network paper. The result show that the slope of efficiency degradation as a function of luminosity is significantly improved in de-noised data sample. It is worth noting that at $90~nA$ the de-noised data sample track reconstruction efficiency is same as for the $45~nA$ when reconstructing raw data sample (without de-noising). This implies that experiment can run at effectively at $90~nA$, collecting data twice faster, while maintaining the same track reconstruction efficiency.

\subsection{Physics Impact}

The processed data was also evaluated to extract physics observables from both data sample to discern the impact on physics for de-noising algorithm. As mentioned before the data selected from Pythia simulation was for the final state $H(e,e^\prime\pi^+\pi^-pX)$. From this sample the missing mass distribution of $H(e,e^\prime\pi^+\pi^-X)$ is analyzed which should show proton mass where the selected reaction was inclusive $\rho$ meson production and some background (above proton mass) where other reactions (with missing neutral particles)  are selected.

\begin{figure}[!h]
\begin{center}
 \includegraphics[height=3.0in]{images/figure_phys_scan.pdf}
 \includegraphics[height=3.0in]{images/figure_phys_conv.pdf}
\caption {Number of reconstructed protons from missing mass of $H(e \rightarrow e^\prime \pi^+\pi^-)$ for background merged files for  $5~nA$, $45~nA$, $95~nA$ and $150~nA$ respectively. The number of protons reconstructed by conventional algorithm after background merging is shown on the top row, and reconstruction after  de-noising drift 
chamber data on the bottom row.}
 \label{physics::conv_dn}
 \end{center}
\end{figure}

On Figure~\ref{physics::conv_dn} the results of analysis are shown, where the missing mass distribution $H(e,e^\prime\pi^+\pi^-)X$ is plotted for both background merged data (top row) and for background merged de-noised data (bottom row) as a function of incident beam current (luminosity).  The scale on all plots is set to the same heigh to visually observe decrease of number of protons reconstructed. 

%\begin{figure}[!ht]
%\begin{center}
 %\includegraphics[width=3.1in]{images/figure_phys_scan.pdf}
 %\includegraphics[width=2.5in]{images/figure_phys_conv_compare.pdf}
%\caption {Number of reconstructed protons from missing mass 
%of $H(e e^\prime \pi^+\pi^-)$ for background merged files for 
%$5~nA$, $45~nA$, $95~nA$ and $150~nA$ respectively.}
 %\label{physics::count_raw_dn}
 %\end{center}
%\end{figure}

On Figure~\ref{physics::conv_dn} (a) the dependence is plotted for both data samples, where the points represent number of events under the proton peak normalized to the number of protons reconstructed by the tracking algorithm before background merging procedure (shown on Figure~\ref{physics::count_raw_dn} in the first column).
It is evident from the figure that number of reconstructed protons in the de-noised data at $45~nA$ is $37\%$ larger, and the number of reconstructed protons at $95~nA$ in de-noised data sample is $2\%$ larger than in $45~nA$ background merged files. This result has significant implications on future experiments, since the data can be collected much fasted to reach the required statistical significance for given physics program while saving significant amount of money in accelerator operation costs.






\section{Data Analysis of De-Noising data with AI assistance}

The two data samples, background merged and de-noised, were also processed with new 
reconstruction software, which includes AI assisted track candidate identification~\cite{Gavalian:2020oxg},\cite{Gavalian:2020xmc}. 
The reconstruction software is designed to be able to process data in two parallel branches, where in one 
branch it reconstructs tracks with conventional algorithm where track candidates are identified by fitting all 
combinations of clusters forming a candidate and choosing candidates that pass the ``goodness'' of the fit criteria, 
 and on the second branch AI classifies track from the list of candidates crated from all combinations of clusters 
 forming a track. 
 %This procedure is described in detail in~\cite{Gavalian:2022hfa}. 
 The details on track candidate identification, software implementation and resulting outcome for increased 
 track reconstruction efficiency can be found in~\cite{Gavalian:2022hfa}.
 Two samples were processed and comparison was made between conventional tacking algorithm from raw 
 background merged files, and output of de-noised data sample with and without AI assisted tracking. 

\subsection{Luminosity dependence}

The track reconstruction efficiency was calculated for three samples using Eq.~\ref{eq::eff}.
The results are presented on Figure~\ref{lscan::conv_dn_ai}. It can be seen from the figure that using 
AI assisted tracking on de-noised data sample further improves reconstruction efficiency. The raw background 
merged data sample exhibits tracking efficiency decline of $0.23\%$ per nA, while the combination of de-noising 
and AI assisted tracking reduces this slope to $0.12\%$ per nA (almost factor of 2), resulting in efficiency of $0.86\%$ at
 beam current $150~nA$ compared to $0.88\%$ at $45~nA$ beam current. 

\begin{figure}[!h]
\begin{center}
 \includegraphics[width=3.1in]{images/figure_lscan_pos_ai.pdf}
 \includegraphics[width=3in]{images/figure_lscan_neg_ai.pdf}
\caption {Tracking efficiency as a function of luminosity (beam current) for positive (a) and negative particle (b).  The efficiency is shown for
conventional algorithm running on background merged files (diamonds), and on files with merged background then de-noised with AI (circles).}
 \label{lscan::conv_dn_ai}
 \end{center}
\end{figure}

This is significant improvement in tracking efficiency when using both AI assisted tracking with de-noising for beam current 3 times higher than current data collecting conditions. 

\subsection{Physics Impact}

Further the physics impact was studied for de-noised data sample processed with AI assisted tracking. Same data 
sample was used in this studies with selected $H(e,e^-\pi^+\pi^-p)X$ event from Pythia simulations, and analyzed for
missing mass of $H(e,e^-\pi^+\pi^-)X$, where number of protons were extracted from under the missing mass. 


\begin{figure}[!h]
\begin{center}
 \includegraphics[height=3.1in]{images/graph_mxepipi_dn_ai.pdf}
 \includegraphics[height=3.1in]{images/plots_mxepipi_dn_ai.pdf}
\caption { a) Number of reconstructed protons from missing mass of $H(e \rightarrow e^\prime \pi^+\pi^-)X$ 
for background merged data set reconstructed with conventional tracking (squares) compared to de-noised data sample 
reconstructed with conventional algorithm (diamonds) and de-noised data sample reconstructed with AI assisted tracking 
algorithm (triangles)  for $45~nA$, $95~nA$ and $150~nA$. b), c) and d) reconstructed missing mass distributions for 
background merged data set reconstructed with conventional tracking (filled histogram) and de-noised data sample 
reconstructed with AI assisted algorithm (solid line histogram). Missing mass distribution for data sample before 
background merging ($0~nA$) is shown (circles) for reference. }

 \label{physics::conv_dn_ai}
 \end{center}
\end{figure}

The distributions of missing mass spectra are shown In Figure~\ref{physics::conv_dn_ai} for different beam current backgrounds.
In b), c) and d) the missing mass distributions are shown for background merged data samples processed with conventional algorithm 
(filled histogram) and reconstructed missing mass after data de-noising and reconstructing with AI assisted tracking (line histogram).
The graphs (circle symbol) on all three plots show missing mass distribution reconstructed from generated data sample 
before any background is added for reference. In Figure~\ref{physics::conv_dn_ai} a) the summary of the studied data samples 
are presented. The background merged data samples analyzed with conventional tracking algorithm (squares) show sharp decline in
number of reconstructed protons under the missing mass peak. Pre-processing data with de-noising auto-encoders and processing
with conventional algorithm (diamonds) improves the physics outcome due to improved single track efficiency. The biggest improvement
comes from using AI assisted track classification software after de-noising the drift chamber data (triangles). 

%The top row of plots shows missing mass distributions for different backgrounds reconstructed by conventional tracking algorithm. On the bottom row the distributions reconstructed from de-noised data sample are shown, with overlaid histograms (red) of AI assisted reconstruction. On Figure~\ref{physics::conv_dn_ai} a) the summarized analysis of missing mass distributions are shown where number of reconstructed protons are plotted for all three reconstruction scenarios. It is evident that adding AI assistance for track classification further improves physics outcome from data processing. It can be seen that for $95~nA$ data sample, the number of reconstructed protons with de-noising and AI assisted tracking is $14\%$ higher than number of protons from background merged file reconstructed using conventional (no AI involved) code.

\section{Discussion}

Studies with simulated data indicate that using de-noising auto-encoders significantly improves the performance
of conventional tracking algorithm (Figure~\ref{physics::conv_dn} a). Further improvements come from using
already established AI assisted track classifier network with de-noised data (Figure~\ref{physics::conv_dn_ai} a).

It is evident from these studies that analysis of existing data can benefit from new approach to tracking by increase 
of statistical significance of physics observables. The numbers for reconstructed protons for each background setting 
and method of track reconstruction are summarized in Table~\ref{table:summary}. Using de-noising and AI assisted 
tracking the statistics (in this particular case of three detected particles) increases by $26\%$ when using AI assisted 
tracking with de-noised data.

%As can bee seen in Figure~\ref{physics::conv_dn_ai} the de-noising the raw data significantly improves 
%track reconstruction efficiency. Addition of AI classifier further improves reconstruction efficiency and
%makes it possible to operate CLAS12 detector at $120~nA$ with same efficiency of nucleon final state 
%reconstruction as for conventional tracking algorithm at $45~nA$. The missing mass distributions were 
%analyzed to extract the number of event under the missing mass peak for each data sample (method of 
%reconstruction and incident beam current combinations). The summary of the study is shown in 
%Table~\ref{table:summary}.

\begin{table}
\begin{center}
\begin{tabular}{l|ccc}
Stats & Conventional & De-noised & De-noised + AI CL \\
\hline
 nucleons (45 nA)  & 27225 &  30576 & 34277 \\
 nucleons (95 nA)  & 17125 & 23845 & 29428 \\
 nucleons (150 nA) &  1576 & 17018 & 23601 \\
\hline
\hline
ratio to conventional 45nA & 1.0 & 1.12 & 1.26 \\
ratio to conventional 95nA & 1.0 & 1.39 & 1.72 \\
ratio to conventional 150nA & 1.0 & 10.80 & 14.97 \\
\end{tabular}
\end{center}
\caption{Number of extracted nucleon count from missing mass distribution for different beam currents
and different reconstruction methods. The bottom of the table presents ratio of number of nucleons for
different methods to the number for conventional tracking algorithm at $45~nA$ for all incident beam currents.}
 \label{table:summary}
\end{table}

Increase in statistics for existing data sets is exciting, however the study suggests even more benefits for 
CLAS12 detector.
From table is can be seen that taking data with $95~nA$ beam current leads to more events in the missing 
mass peak than running at $45~nA$ and reconstructing with conventional tracking algorithm. And conducting
experiment with $95~nA$ incident beam energy will take twice shorter to accumulate same number of events.
Even though number of reconstructed nucleons is bigger when running at $45~nA$ and using improved 
tracking (including AI de-noising and AI classifier), the argument can be made that collected statistics at 
$95~nA$ (because of rate of interactions at higher incident beam current) will lead to more physics relevant
statistics even with slightly lower track reconstruction efficiency.
Second half of the Table~\ref{table:summary} shows ratio of number of nucleon reconstructed under the 
missing mass peak for different beam currents and algorithms used. It can bee seen that at with increased 
beam current the de-noiser gain over the conventional algorithm is exponentially increasing, indicating that 
de-noiser is very efficient in isolating hits that potentially belong to a ``true'' track candidate.
This study suggest that augmenting tracking algorithms with artificial intelligence opens the possibility
of conducting experiments at higher luminosity collecting larger data samples for physics reactions and in 
shorter time. This will definitely affect the estimation of experimental running conditions for CLAS12 
detector for future experiments.



%\section{Analysis of experimental data}

We have verified that de-noising of background merged files significantly improves the track reconstruction efficiency.
It was shown that using AI assisted version of reconstruction software further improved the number of particles reconstructed for given reaction. 
The further checks experimental data collected with $45~nA$ incident beam energy was processed using de-noising software and then was processed with CLAS12 data reconstruction program. The same reaction $H(e,e^\prime\pi^+\pi^-)X$ was selected (for consistency) from experimental data and missing mass distributions were plotted for raw data, de-noised data reconstructed with conventional tracking algorithm and de-noised data reconstructed with AI assisted tracking algorithm.
The results are shown on Figure~\ref{denoise:exp_data}


\begin{figure}[!h]
\begin{center}
 \includegraphics[width=3.1in]{images/figure_denoise_expdata.pdf}
\caption {Tracking efficiency as a function of luminosity (beam current) for positive (a) and negative particle (b).  The efficiency is shown for
conventional algorithm running on background merged files (diamonds), and on files with merged background then de-noised with AI (circles).}
 \label{denoise:exp_data}
 \end{center}
\end{figure}


\begin{table}[!h]
\begin{center}
\begin{tabular}{lccc}
Method & protons count & ratio &  MC ratio \\
\hline
Conventional & 41 & 1.00 & 1.00 \\
De-Noised & 59 & 1.43 & 1.37 \\
De-Noised with AI assisted & 70 & 1.71 &1.65 \\
\hline
\end{tabular}
\end{center}
\end{table}



%\section{Analysis of Track Reconstruction with AI}

After the implementation of track identification service in CLAS12 reconstruction software the outputs
from conventional tracking algorithm and artificial intelligence assisted tracking algorithm were analyzed
event by event to ascertain improvements of tracking. 
 
 \subsection{Particle Reconstruction efficiency}
 
 The Neural Network for track classification was trained on experimental data after it was processed with conventional tracking 
 reconstruction. Track that have "good" fit quality and were tracked back to the target were used as a training sample for both 
 MLP classifier and Auto-Encoder corruption fixing network. For detailer analysis of tracking reconstruction performance with and without 
 assistance from artificial intelligence we processed one run at nominal luminosity (45 nA) compare performances.
 
 %The efficiency of track reconstruction was obtained for separate track topologies (6 super-layer and 5 super-layer).
 \begin{figure}[!h]
\begin{center}
% \includegraphics[width=2.0in]{images/pos_theta_5SL.png}
  \includegraphics[width=5.0in]{images/summary_5SL_6SL_neg.pdf}
\caption { }
 \label{track:efficiency}
 \end{center}
\end{figure}

The results are shown on Figure~\ref{track:efficiency}, where dependence of number of reconstructed negatively charged 
 tracks are shown as a function of particle momentum (top row), polar angle in laboratory frame (bottom row) and interaction
 vertex (middle row). The reconstructed distributions from conventional tracking are plotted with filled histograms and the
 tracks reconstructed by using assistance from AI are plotted with solid lines. As can be seen from the figure there is very big gain 
 in number of reconstructed tracks with 5 super-layer configuration compared to full 6 super-layer tracks. Typically for nominal 
 45 nA experimental data increase track efficiency for 6 super-layers tracks averages about $3\%-6\%$, while for 5 super-layer
 tracks the increase is in the order of $70\%-120\%$. In normal data reconstruction tracks that are identified with 5 super-layers
 usually comprise about $10\%$ of all reconstructed tracks, and significant increase in identification of such tracks leads to 
 overall tracking efficiency increase of $12\%-15\%$. 
 
 \begin{table}[!h]
 \begin{center}
 \begin{tabular}{|l|c|c|c|c|}
 \hline
 Track Configuration & Conventional & AI Assisted & Gain & Relative \\
 \hline
 \hline
 6 Super-Layer & 242,145 & 256,175 & 14030 & 1.0579 \\
 5 Super-Layer & 24,155 & 52,839 & 28684 & 2.1875 \\
 All & 267,339 & 309,058 & 51719 & 1.1561 \\
 \hline
 \end{tabular}
 \end{center}
 \caption{Summary of reconstructed tracks and gain with assistance from Artificial Intelligence.}
 \label{tbl:summary}
 \end{table}
 
The comparison of 5 super-layer and 6 super-layer track statistics and their relative gain is summarized in Table~\ref{tbl:summary}.
As can be seen from the table the gain in only 6 super-layer tracks is about $5.7\%$ but with significant gain in 5 super-layer tracks 
the overall gain in reconstructed tracks elevates to $>15\%$. These results are intuitive since track candidates composed of 5
super-layers with the same number of clusters in each super-layer are significantly higher than 6 super-layer track candidates, and 
it our tests AI performs better in choosing right combination with increasing combinatorics.
 
\subsection{Luminosity Dependence}

Track reconstruction efficiency increased with AI assisted tracking, since AI can better identify tracks
from pool of candidates. One would expect that if the number of combinations decrease efficiency 
of conventional track selection algorithm should approach efficiency of AI assisted track identification,
similarly, when number of combination increases the advantage of AI over the conventional algorithm should
increase. Based on this we expect AI to perform better in higher background settings. To evaluate AI assisted
tracking efficiency dependence on background we analyzed several different runs that were taken in different 
conditions (i.e. beam current) ranging from $5~nA$ to $70~nA$. To measure tracking efficiency we first calculated
number of electrons ($N_e$) detected in the data sample analyzed (typically one run) and then number of positive and negative
 hadrons that were detected with the electron inclusively ($N_{h^+e}$ and $N_{h^-e}$ respectively).

Then the efficiency for the data set was calculated as:

\begin{equation}
L_t^+ = \frac{N_{h^+e}}{N_e} , L_t^- = \frac{N_{h^-e}}{N_e} 
\end{equation}

where $L_t^+$ is the efficiency of positive particles and $L_t^-$ is the efficiency of negatively charged particles respectively. 
In order to estimate the charged particle reconstruction efficiency as a function of the beam current, the multiplicity, $L_t^{+/-}$, is fitted with a linear function:
\begin{equation}
L_t^{+/-} = a + c\times I 
\end{equation}

Here $a$ and $c$ are the fit parameters and $I$ is the beam current. Then it was assumed that the reconstruction efficiency, $E=1$ at $I=0$ nA:

\begin{equation}
E^{+/-} = 1 + b \times I 
\end{equation}

with $b=\frac{c}{a}$. The slope parameter b is the rate of the reconstruction inefficiency as a function of the beam current \cite{Stepanyan:2020bg}.
%The all points were 
%fitted with linear function $L=a+bx$, where $a$ is the intercept 
 
 \begin{figure}[!ht]
\begin{center}
 \includegraphics[width=3.0in]{images/lumi_scan_positive.pdf}
 \includegraphics[width=3.0in]{images/lumi_scan_negative.pdf}
\caption {Tracking efficiency for positively and negatively charged particles as a function of beam current (luminosity).  Conventional algorithm 
track reconstruction efficiency (diamonds) is compared to AI assisted track reconstruction efficiency (circles). }
 \label{lumi:scan}
 \end{center}
\end{figure}

The comparison of tracking efficiency as a function of beam current (luminosity) can be seen on Figure~\ref{lumi:scan} where $E^{+/-}$ are shown for positively and negatively charged particles separately. As can be seen from the figure AI assisted tracking performs significantly better for any given luminosity (beam current) and the decrease of efficiency is much slower as function of luminosity, $0.22\%$ per nA versus $0.40\%$ per nA for conventional tracking. This is expected and consistent with assumption that with increased combinatorial background (increased number of track candidates to consider) , AI performs better in choosing best track candidate. We established that AI assisted
tracking leads to more tracks reconstructed for any given beam current setting, next thing to check is what is the impact of increase track reconstruction
efficiency on physics analysis, and if there is increase in physics outcome for the CLAS12 experimental setup.

\subsection{Physics Impact}

To measure practical implications of track reconstruction efficiency improvement on physics outcome we analyzed 
two event topologies with two particle and three particles in the final state respectively. The data for analysis were 
taken with $10.5~GeV$ electron beam incident on $20~cm$ liquid hydrogen target, with the beam current of $45~nA$
(typical for CLAS12 experimental running). We selected events where an electron was detected in the forward detector 
then isolated events where there was an additional negatively charge pion ($\pi^-$) along with electron and no other 
charged particle, and the second topology required two pions along with electron, one positively charged and one 
negatively charged. The two chosen topologies are denoted by $H(e,e'\pi^-X)$ and $H(e,e'\pi^+\pi^-X)$. In both cases 
there is a visible peak of a missing proton that we can use to measure impact of efficiency on physics outcome. 

 \begin{figure}[!ht]
\begin{center}
 \includegraphics[width=6.0in]{images/physics_results.pdf}
\caption {Architecture of corruption fixing Auto-Encoder.}
 \label{physics:outcome}
 \end{center}
\end{figure}

The distributions of missing mass for both final state topologies are shown on Figure~\ref{physics:outcome}, the plots 
on the top row are missing mass of $H(e,e'\pi^-X)$ and $H(e,e'\pi^+\pi^-X)$, where the filled histogram is calculated from 
particles reconstructed by conventional tracking algorithm, and the histogram with black outline are same distributions 
calculated from particles that were reconstructed using suggestion from Artificial Intelligence. As can be seen from the figure 
there is significant increase in number of events under the missing proton peak (at mass value $0.938~MeV$) for AI assisted
tracking. The ratios of two histograms (AI assisted divided by Conventional) can be seen on the bottom row of 
Figure~\ref{physics:outcome}. As can be seen from the figure the increase in statistics is uniform over the whole range of the 
missing mass indicating no systematic abnormalities for AI assisted tracking. The ratio also indicates that there is an increase 
for number of event under the peak for proton, about $15\%$ for $H(e,e'\pi^-X)$ final state and $30-35\%$ for the $H(e,e'\pi^+\pi^-X)$
final state. Further studies show that increase in statistics for different final states increase with increase of beam current (luminosity) 
which is consistent with our studies of increased efficiency of single particle reconstruction. 


\section{Summary}

In this paper we investigated results of analysis of experimental data from CLAS12 detector reconstructed with assistance of Artificial Intelligence
to identify tracks from the hits in Drift Chambers. This work is based on two Neural Networks developed to classify track candidates from
given cluster combinations \cite{Gavalian:2020oxg} and to identify missing cluster positions in tracks that do not have complete 6 cluster configuration \cite{Gavalian:2020xmc}. After implementing these networks into CLAS12 reconstruction workflow, the AI was able to identify "good" track candidates 
and pass them to tracking code to be analyzed parallel to conventional algorithm that chooses "good" track candidates iteratively considering all possible combinations. 
Our studies showed that AI assisted tracking performs better than conventional track identification algorithm, and leads to track reconstruction efficiency increase of $15\%$ for nominal experimental running conditions (beam current 45 nA). The AI also performs better with increasing background (i.e. with increased incident beam current) and improves the efficiency loss from $0.44\%$ per nA to $0.24\%$ per nA.
This increased track reconstruction (identification) efficiency directly impacts the outcome of physics analysis where it increases statistics 
for physics reactions for $15\%-35\%$ depending on how many particles are detected in the final state and the topology of the reaction. This has big implication on experimental running conditions, since with increased efficiency required statistical significance of the experiment can be reached in shorter time by running at higher beam current (luminosity). Already collected experimental data be re-processed with the AI assisted tracking
code which can increase the statistics for analyzed data up to $35\%$. Both, future experiments and already completed experiment will benefit 
from this novel development.

Another important outcome of this development was reduction in data processing times, since track candidates were identified by Artificial Intelligence 
there were fewer marginal quality tracks picked to be analyzed and then later dropped due to non convergence of Kalman-Filter, and this leads to tracking code speed-up of $35\%$.

Overall we identified that Artificial Intelligence approach to assisting tracking codes is a good approach, it leads to improvements in code speed and 
efficiency of track reconstruction. Another important aspect of using Artificial Intelligence is that is leads to very small and simple code base, comprised of composing track candidates and feeding them to neural network, and what's also important it keeps improving with constant training with new data.
We intend to continue this development in extending the approach to other tracking detectors of CLAS12, and possibly try to adopt  our approach for other experimental detector setups at Jefferson Lab.



\section{Discussion}

The HiPO data format was developed for CLAS12 to serve as a data format for the entire life cycle of the experimental data. The CLAS12 reconstruction uses it for data processing and production data storage. Petabytes of data are stored in HiPO and have successfully been used by the collaboration to run their physics analysis workflows for the past 7 years. Since the introduction of the single data format, standard tools were developed for data file manipulations, such as data merging, filtering, and selective reduction, eliminating the need for users to write and maintain codes for the most common data manipulation tasks. There are tools (graphical and text-based) to browse the files and display the content of each event for debugging.
The recent development of columnar data storage can now also be used for storing columnar data from physics analysis. The tests show that it provides exceptional performance in data analysis, surpassing the more established data formats (ROOT and Parquet) currently used in the High Energy and Nuclear Physics fields.
Future improvements in binding to other languages, such as Python and Julia, will extend the usability of the data format for analysis.





\newpage

\section{Acknowledgments}

This material is based upon work supported by the U.S. Department of Energy, Office of Science,
Office of Nuclear Physics under contract DE-AC05-06OR23177, and NSF grant no. CCF-1439079 and
the Richard T. Cheng Endowment. This work was performed using the Turing and Wahab computing
clusters at Old Dominion University.
 
\newpage
\bibliography{references}
\bibliographystyle{ieeetr}

\end{document}
